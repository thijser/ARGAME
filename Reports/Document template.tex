\documentclass[11pt] {article}

\usepackage{algorithmicx}
\usepackage{algorithm}
\usepackage{algpseudocode}

\usepackage{amsmath}
\usepackage{amssymb}
\usepackage[section] {placeins}
\usepackage[verbose, a4paper, margin=0.5in] {geometry}

\setlength{\parindent}{0pt}
\setlength{\parskip}{5pt plus 2pt minus 1pt}

\title{Plan van Aanpak}
\author{Collaborative augmented reality game}
\date{}

\begin{document}
\maketitle

\section{Game concept}

uitleg van spiegel game idee, precieze rollen van remote en local spelers, basis concept en hoe het later uitgebreid kan worden

\section{Aanpak}

technische details, zoals gebruik van ar toolkit en oculus rift + camera's

ook belangrijk zijn de afspraken, bijv. 1x per week een meeting/email met de coach

\section{Planning}

overzicht van taken en doelen van elke sprint, bijv. basis AR demo met kaarten en 3D spiegels na eerste 2 weken. we moeten tijdens het ontwikkelen terug kunnen verwijzen hierna om te laten zien dat we op schema zitten

\end{document}