\documentclass[]{article}


\usepackage[a4paper, total={7in, 10in}]{geometry}


\begin{document}

\section*{General information}
\textbf{Project title:} Collaborative AR mirror game\\
\textbf{Name of the client organization:} TU Delft\\
\textbf{Date of the final presentation:} July 3, 2015\\
\textbf{Final report:} TBA.\\

\section*{Description of the project}
While augmented reality research has grown into a mature field over the last years, the aspects of situational awareness and presence of augmented reality (AR) are still quite open research topics. This project is about designing and implementing a collaborative game to explore the different perception of situational awareness, presence and workload in a physical and an AR environment. The game is to be employed as an approximation of collaboratively solving complex problems, as they occur in crime scene investigation when using virtual co-location, i.e. expert remote crime scene investigators to guide local investigators in AR to collaboratively analyse the crime scene. The goal of the game is to jointly guide a laser beam from its source to its target, mainly by using mirrors to deflect the laser beam to its goal. The rules of the game are designed to require collaboration of the different players. While the shared goal is guide the laser from source to target, remote players can only rotate mirrors, not move them, and vice versa for local players. Individual expertise is mimicked by the ability to either move or rotate mirrors, and shared expertise is mimicked by being able to get to the goal using movement and rotation of markers. These rules are to be considered as initial set of rules, but can be altered to create more intensive game play experiences. It has, though, to be possible to play the game with at least three players in two environments: 1. Physical environment: all players are present in the same location and collaboratively build a tower using physical blocks. 2. AR environment: at least two players are present at the same location (physically co-located). At least one player is physically remote but virtually co-located. The game has to be developed in Unity3D, optionally using DECLARE (DistributEd CoLlaborative Augmented Reality Environment (DECLARE)) being developed at the TPM faculty. DECLARE offers AR marker recognition, natural feature tracking, free hand interaction and basic support for shared data. On the hardware side, AR optical see-through devices such as the META One are provided for the local players. For the remote player(s), Oculus Rift in combination with a Leap Motion might be used, but this is optional. 

\section*{Members of the project team}
\textit{Name:} Thijs Boumans\\
\textit{Interests:}\\
\textit{Contribution and role:} front-end developer\\
\\
\textit{Name:} Patrick Kramer\\
\textit{Interests:}\\
\textit{Contribution and role:} lead software architecture\\
\\
\textit{Name:} Alexander Overvoorde (overv161@gmail.com)\\
\textit{Interests:} Computer Graphics, Software Engineering\\
\textit{Contribution and role:} remote player visualization, game mechanics, server computer vision\\
\\
\textit{Name:} Tim van Rossum (trvanrossum@gmail.com)\\
\textit{Interests:} Algorithm Design, Software Engineering \\
\textit{Contribution and role:} Final Report curator, level designer, co-lead software architecture\\

\section*{Coach and client}
\textbf{Coach:} Rafa\"el Bidarra, Computer Graphics, r.bidarra@tudelft.nl\\
\textbf{Client:} Stephan Lukosch, Multi-Actor Systems, s.g.lukosch@tudelft.nl
\end{document}
