\documentclass[]{article}


\usepackage[a4paper, total={7in, 10in}]{geometry}


\begin{document}

\section*{General information}
\textbf{Project title:} Collaborative AR mirror game\\
\textbf{Name of the client organization:} TU Delft\\
\textbf{Date of the final presentation:} July 3, 2015\\
\textbf{Final report:} TBA.\\

\section*{Description of the project}
Although augmented reality research has grown into a mature field over the last
years, the aspects of situational awareness and presence of augmented reality
are still open research topics. The mirrors AR game is a collaborative game that
can be used to explore different perceptions of situational awareness and the
effect on solving spatial puzzles. \\
\\
Players have to guide one or more laser beams from emitters to targets by
placing mirrors inside a level. Each level poses new challenges by including
walls and optical components like beam splitters. \\
\\
There are two kinds of players: local and remote. Local players wear see-through
AR glasses and see the level projected onto a surface in a room. They can walk
around and place so-called markers on the table to indicate where in the level
they want to place a mirror. Remote players are not physically co-located and
see just the level and mirrors on their computer screen. However, only they have
the power to rotate mirrors and a good overview of the entire level. The players
will need to cooperate to get all of the mirrors in the right places with the
correct orientation to complete the level. \\
\\
The game can support any number of players as long as there's at least one local
and one remote player. The mechanics and restrictions of the game can be
controlled to research the effects on collaboration. For example, the viewpoint
of local players can be shown or hidden to remote players.

\section*{Members of the project team}
\textit{Name:} Thijs Boumans\\
\textit{Interests:} Computer Graphics, Algorithm Design \\
\textit{Contribution and role:} Front-end Developer \\
\\
\textit{Name:} Patrick Kramer (ptrck.krmr@gmail.com) \\
\textit{Interests:} Software Engineering, Quality Assurance \\
\textit{Contribution and role:} Lead Software Designer, AR Projection Mechanics \\
\\
\textit{Name:} Alexander Overvoorde (overv161@gmail.com)\\
\textit{Interests:} Computer Graphics, Software Engineering\\
\textit{Contribution and role:} Remote Player Visualization, Game Mechanics, Server Computer Vision\\
\\
\textit{Name:} Tim van Rossum (trvanrossum@gmail.com)\\
\textit{Interests:} Algorithm Design, Software Engineering \\
\textit{Contribution and role:} Final Report curator, Level Designer, Co-lead Software Designer \\

\section*{Coach and client}
\textbf{Coach:} Rafa\"el Bidarra, Computer Graphics, r.bidarra@tudelft.nl\\
\textbf{Client:} Stephan Lukosch, Multi-Actor Systems, s.g.lukosch@tudelft.nl
\end{document}
