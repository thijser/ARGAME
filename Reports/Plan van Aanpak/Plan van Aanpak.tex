\documentclass[11pt] {article}

\usepackage{algorithmicx}
\usepackage{algorithm}
\usepackage{algpseudocode}

\usepackage{amsmath}
\usepackage{amssymb}
\usepackage[section] {placeins}
\usepackage[verbose, a4paper, margin=0.5in] {geometry}

\setlength{\parindent}{0pt}
\setlength{\parskip}{5pt plus 2pt minus 1pt}

\title{Product planning}
\author{Thijs Boumans\\ Patrick Kramer\\ Alexander Overvoorde\\ Tim van Rossum}
\date{}

\begin{document}
\maketitle

\section{Game concept}

The goal of the game is to solve a puzzle by controlling laser beams using mirrors
in such a way that a predefined target is hit. The game can be played by 
one or more local players and one or more remote players.

There are cards present for the local players that represent mirror bases. These 
must be placed on the table, which will be the locations for the mirrors. The local
players will be able to see the mirrors they place through the use of AR technology.
Each of the local players will only be given a few of the mirror bases needed to solve
the puzzle, and as such solving the puzzle requires cooperation from all local players.

The remote players can also see the placed mirrors, and can rotate them to influence 
the path of the laser beam(s). The remote players is also the only one who can see the
obstacles that are predefined for that particular puzzle. Only by cooperation between 
local players (who can only move the mirror bases) and remote players (who can only 
rotate them) it becomes possible to hit the target and as such solve the puzzle.

The game provides various different types of mirrors with different properties, 
allowing for more complex puzzles. One example of such a mirror is a colored mirror, 
and then require the target is hit with the right (combination of) colors. Another 
way to make puzzles more complex is requiring that the players combine beams together
to create more powerful beams. Other optical components like beam splitters can
also be introduced.

\subsection{Requirements}

\paragraph{Must haves}
\begin{itemize}
	\item A light source and a target must be visible when the game starts.
	\item A light source must emit a light beam in a predefined direction.
	\item A light beam must reflect against a mirror when it hits one.
	\item A local player must be able to see mirrors over the cards using AR technology.
	\item A local player must be able to move the mirrors by moving the cards.
	\item A remote player must be able to see the mirrors in the same positions and 
		  orientations as the local players.
	\item A remote player must be able to rotate the mirrors.
\end{itemize}

\paragraph{Should haves}
\begin{itemize}
	\item TODO
\end{itemize}

\paragraph{Could haves}
\begin{itemize}
	\item TODO
\end{itemize}

\paragraph{Won't haves}
\begin{itemize}
	\item TODO
\end{itemize}

\section{Approach}

technische details, zoals gebruik van ar toolkit en oculus rift + camera's

test coverage, scrum, testen met gebruikers

ook belangrijk zijn de afspraken, bijv. 1x per week een meeting/email met de coach

\section{Planning}

The first two weeks represent the research phase. In this phase we will find a suitable augmented reality (AR) library for Unity and prepare an Oculus Rift for AR use with cameras. We'll also design an architecture for the game that covers the marker detection, networking and gameplay mechanics. All of this information will be described in the research report handed in on May 1st.

Main development commences after this phase and is organized in two week sprints. The table below describes the goals per sprint, which will serve as a helpful reference to stay on schedule during development.

\begin{tabular}{|l|c|r|}
	\hline
	Weeks & Deadline & Goal \\ \hline
	4.1 + 4.2 & May 1st & Research report described above \\ \hline
	4.3 + 4.4 & May 15th & ??? \\ \hline
	4.5 + 4.6 & May 29th & ??? \\ \hline
	4.7 + 4.8 & June 12th & ??? \\ \hline
	4.9 + 4.10 & June 26th & ??? \\ \hline
\end{tabular}

\end{document}