\documentclass[11pt] {article}

\usepackage{algorithmicx}
\usepackage{algorithm}
\usepackage{algpseudocode}

\usepackage{amsmath}
\usepackage{amssymb}
\usepackage[section] {placeins}
\usepackage[verbose, a4paper, margin=0.5in] {geometry}

\setlength{\parindent}{0pt}
\setlength{\parskip}{5pt plus 2pt minus 1pt}

\title{Plan van Aanpak}
\author{Thijs Boumans\\ Patrick Kramer\\ Alexander Overvoorde\\ Tim van Rossum}
\date{}

\begin{document}
\maketitle

\section{Concept}
This section covers the concept idea of the gameplay, and the accompanying
requirements.

\subsection{Gameplay}
The goal of the game is to solve a puzzle by controlling laser beams using mirrors
in such a way that a predefined target is hit. The game can be played by 
one or more local players and one or more remote players.

There are cards present for the local players that represent mirror bases. These 
must be placed on the table, which will be the locations for the mirrors. The local
players will be able to see the mirrors they place through the use of AR technology.
Each of the local players will only be given a few of the mirror bases needed to solve
the puzzle, and as such solving the puzzle requires cooperation from all local players.

The remote players can also see the placed mirrors, and can rotate them to influence 
the path of the laser beam(s). The remote players is also the only one who can see the
obstacles that are predefined for that particular puzzle. Only by cooperation between 
local players (who can only move the mirror bases) and remote players (who can only 
rotate them) it becomes possible to hit the target and as such solve the puzzle.

The game provides various different types of mirrors with different properties, 
allowing for more complex puzzles. One example of such a mirror is a colored mirror, 
and then require the target is hit with the right (combination of) colors. Another 
way to make puzzles more complex is requiring that the players combine beams together
to create more powerful beams. Other optical components like beam splitters can
also be introduced.

\subsection{Requirements}

\paragraph{Must haves}
\begin{itemize}
	\item A light source, a target and zero or more blocks must be visible when the game starts.
	\item A light source must emit a light beam in a predefined direction.
	\item A light beam must reflect against a mirror when it hits one.
	\item A light beam must stop when it hits a block
	\item A local player must be able to see mirrors over the cards using AR technology.
	\item A local player must be able to move the mirrors by moving the cards.
	\item A remote player must be able to see the mirrors in the same positions and 
		  orientations as the local players.
	\item All players must be able to see the light beam, the light source, the target and the blocks.
	\item A remote player must be able to rotate the mirrors.
\end{itemize}

\paragraph{Should haves}
\begin{itemize}
	\item There should be elevations of mirrors, to allow light of a lower elevation to hit the target on a higher elevation, or the other way around
	\item The light beams should be colored (after a certain level), and only light beams of a certain color can suffice in hitting the target.
	\item There should be combiners (think AND- or OR-gates) that combine light and cause it to travel to the target.
	\item There should be color combiners, to allow for a broader variety of color-specific targets.
\end{itemize}

\paragraph{Could haves}
\begin{itemize}
	\item The game could have an infinite amount of levels.
	\item The game could have random level generation according to maximum difficulty settings.
	\item The game could give hints if the players are stuck.
\end{itemize}

\paragraph{Won't haves}
\begin{itemize}
	\item The game won't be playable with only remote players.
	\item The game won't be playable with only local players.
	\item The game won't be playable on Android or iOS devices.
\end{itemize}


\section{Aanpak}

technische details, zoals gebruik van ar toolkit en oculus rift + camera's

ook belangrijk zijn de afspraken, bijv. 1x per week een meeting/email met de coach

\section{Planning}

overzicht van taken en doelen van elke sprint, bijv. basis AR demo met kaarten en 3D spiegels na eerste 2 weken. we moeten tijdens het ontwikkelen terug kunnen verwijzen hierna om te laten zien dat we op schema zitten

\end{document}