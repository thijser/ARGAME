\section{Approach}

This section covers development details that aren't directly related to the
gameplay itself, such as software engineering practices and guidelines about
client meetings.

\subsection{Technical details}

Although AR glasses have been provided to us, their field-of-view is very
limited and is not suitable for most of our concepts. We experienced ourselves
that the limited fov causes a lot of problems. Instead, we're going to try to
use an Oculus Rift and cameras to create our own high fov augmented reality
glasses.

The Unity game engine is going to be used to render the in-game objects like
mirrors and laser beams over the camera image.

{\color{red} TODO: Find suitable AR library and write about it here}

\subsection{Software engineering}

Using the Unity game engine means that we'll use a loosely coupled
component-based architecture from the start. Unity has built-in unit and
integration testing systems, which we'll make extensive use of in development.
We'll also maintain UML diagrams of the architecture to keep an overview of how
everything works, and to plan integration of new features.

We'll make use of the Scrum software development methodology to plan new
features and to help stick to the schedule. As is expected, there will be a
playable demo at the end of each weekly sprint. As soon as the game has all the
basic ingredients to be fun to play, we'll find users to play test on a regular
basis and collect feedback from them to improve the gameplay.

\subsection{Guidelines}

Here is a list of rules that help prevent problems during development.

{\color{red} NOTE: These are just ideas right now}

\begin{itemize}
	\item Meet with the client and coach every week to show a working demo
	\item Add tests as soon as new methods are added to verify that they work
	\item Have integration tests for common scenarios to reduce the need for user testing
\end{itemize}