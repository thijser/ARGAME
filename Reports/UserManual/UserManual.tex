\documentclass{report}
\usepackage{graphicx}

% Title Page
\title{AR Mirror Game user manual}
\author{Thijs Boumans \and Patrick Kramer \and Alexander Overvoorde \and Tim van Rossum}

\begin{document}
\maketitle

\section*{Acquiring the source code}
	The source code can be downloaded using git from the following repository:
	
	\verb#git@github.com:thijser/ARGAME.git#

\section*{Building the server}
	The server code is in the \verb#MirrorServer# directory and includes a Qt
	Creator project to build it. It depends on the latest version of OpenCV, which
	needs to be built manually using Visual Studio.
	
	After building it, the "Projects" menu needs to be opened and the
	\verb#OPENCV_HOME# variable needs to be set to a directory with the following
	subdirectories:

	\begin{itemize}
	    \item \verb#include# - Include directories (\texttt{.h} files)
	    \item \verb#bin# - OpenCV libraries (\texttt{.dll}, \texttt{.so} or \texttt{.dylib} files, depending on the operating system)
	    \item \verb#lib# - OpenCV library definitions (\texttt{.lib} files)
	\end{itemize}
	
	After running \verb#Build -> Run qmake# the server can then be built using
	\verb#Build -> Rebuild All#. The output files will be placed in the \verb#dist#
	folder in the repository root directory.

\section*{Building the client}
	The client scenes are contained in the \verb#ARGame# directory, under 
	\verb#ARGame\Assets\Scenes#. The scenes can be built using Unity, as they are 
	Unity scenes. It depends on the latest version of Unity (5.1.0, as of writing 
	this manual). Also, because the Meta SDK only works under Windows, local players
	cannot use the application when built for Linux and Mac platforms.
	
	Building it can be done by following these steps:
	
	\begin{itemize}
		\item Go to \verb#File -> Build Settings#
		\item Add the scene that needs to be built (either RemotePlayer.unity or LocalPlayer.unity)
		\item Select the option "PC, Mac \& Linux Standalone" as the platform. Make sure the 
		      target platform is set to Windows, because the Meta does not work on other platforms,
		      as mentioned above.
		\item (Optional) check "Development Build", and afterwards "Script Debugging" if you
			  want to see debugging output from the scripts.
		\item Click "Build", or, if you want to run it immediately, click "Build and Run"
	\end{itemize}

\section*{Starting up the server}
	Before one can start up the server and actually have it do something, the master
	camera needs to be connected to the computer that the server should run on. 
	Then, the user should select whether completely red markers or red-and-yellow
	markers should be detected as corner markers. They should also enter the camera
	device ID (this is system dependent and may change even between different runs on the 
	same system). The server port and camera resolution can also be specified (the default 
	server port is 23369, and it is advised to not change it as that also has to be changed
	client-side in that case). Enabling "Require empty board" will require the board
	to be empty (no markers on the board) before the server is initialized. This is
	used to capture an image of the empty board for remote players.
	
	The server UI also specifies some useful information: it shows the FPS the
	server is recording the board at, the index of the current level, and a debug
	overlay (if enabled) that shows what the camera sees, with marker numbers
	overlaid on the different markers. The server UI is shown in \ref{fig:serverui}
	\begin{figure}[!ht]
	    \centering
	    \includegraphics[scale = 0.5]{MirrorServerUI}
	    \caption{The server interface}
	    \label{fig:serverui}
	\end{figure}
	
	\section*{Joining as a local or remote player, through the build}
	In order to join the game as either a local or remote player by using the built
	executable, one needs to run the executable. The user is then presented with a
	simple UI, which contains a text field to enter the IP or a standard name (like
	"localhost"). One can then choose whether they are a local player or a remote
	player. An image of the UI is shown in \ref{fig:remotelocalui}. Also shown is the
	error message provided when connecting to a host that does not run a mirror server.
	\begin{figure}[!ht]
	    \centering
	    \includegraphics[scale = 0.6]{RemoteLocalUI}
	    \caption{The UI usable for choosing between local and remote player.}
	    \label{fig:remotelocalui}
	\end{figure}
	
	To properly play as a local player, one needs to connect the META One beforehand.
	There is no (foolproof) option right now to check whether or not the META One is
	connected to the device of the player.

\section*{Joining as a remote player, using Unity}
	In order to join the game as an extra remote player through Unity, the corresponding
	scene must be opened. The scene is called \texttt{RemotePlayer.unity}, and holds a 
	single object, the remote controller. This object contains a lot of scripts necessary
	for the functioning of the remote player gameplay.
	
	In order to connect to the server, one has to know the IP address of the
	device that the server runs on. Once known, the server can be accessed by
	entering the IP address into the "Server Address" field of the Client Socket
	component of the remote controller object. The server port is set to a default
	23369, and can be changed, but just like changing the port of the server,
	this is not recommended. A detailed overview of the remote controller object
	and its options can be found in \ref{fig:remotecontroller}
	\begin{figure}[!ht]
	    \centering
	    \includegraphics[scale = 0.6]{RemoteController}
	    \caption{The settings for the remote controller object.}
	    \label{fig:remotecontroller}
	\end{figure}

\section*{Joining as a local player, using Unity}
	In order to join the game as an extra remote player through Unity, the 
	corresponding scene must be opened, and to be able
	to play properly as a local player, the player should also have a META One
	connected to their device. The scene contains a single object, the MetaWorld object. This object
	contains all the settings necessary for the local player gameplay.
	
	In order to connect to the server, one has to know the IP address of the
	device that the server runs on. Once known, the server can be accessed by
	entering the IP address into the "Server Address" field of the Client Socket
	component of the MetaWorld object. Again, it is possible to also change the
	server port, but it is discouraged. The settings of the MetaWorld object
	are shown in \ref{fig:metaworld}
	
	\begin{figure}[!ht]
	    \centering
	    \includegraphics[scale = 0.6]{MetaWorld}
	    \caption{The settings for the MetaWorld object.}
	    \label{fig:metaworld}
	\end{figure}

\section*{Creating a new level and playing it}
	New levels are easily created in Tiled (see report for more details). The size 
	of the level can be chosen to match the available area. For reference, one unit
	is equal to the size of the black and white pattern area on the markers (excluding 
	the green border).  
	
	A level can be created using the tileset and entering tiles 
	from the tileset into the level (a key for which tile is which is easily found 
	on the repository). Then, the level can be exported as an XML file, and saved 
	in the Resources/Levels folder. Note that the default extension of Tiled maps is
	\texttt{tmx}, but the file name needs to end in \texttt{.xml} for Unity to be able
	to find the file.
	
	After saving it as an XML file, the level name (without the extension)
	has to be added to the file \texttt{index.txt} found in the Levels directory of 
	the project's assets. The contents of the index file may be changed to 
	change the levels and order in which they appear in the game. Note that the Unity
	project will have to be rebuilt for the changes to take effect.

\end{document}          
