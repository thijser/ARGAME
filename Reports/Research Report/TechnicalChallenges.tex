\chapter{Technical Challenges} \label{cha:technicalchallenges}
	% Includes the challenges surrounding AR and Networking.
	% Also motivates the choices made.
	This section provides a detailed analysis of the available solutions to the 
	technical challenges mentioned in chapter \ref{cha:analysis}. It compares
	these solutions and provides the solutions that are considered most 
	appropriate to use for this project.
	
	\section{Augmented Reality Hardware} \label{sec:arhardware}
	% Compares various AR hardware devices.
	
	% Hardware devices to consider:
		\subsection{META One} \label{ssec:metaone}
	%   - META One (as indicated by the BEPSys project page)
	%       - Has a limited Field-of-View (around 35 degrees)
	%          - May interfere with the experience of the game.
	%       - Optical see-through glasses means AR works out of the box.
		\subsection{Oculus Rift} \label{ssec:oculusrift}
	%   - Oculus Rift + mounted cameras (http://oculusvr.com/)
	%       - High Field-of-View (100 degrees)
	%       - VR glasses, so need to project the real world using cameras.
	%          - Limited resolution creates blurriness.
	%          - Projection can be done from within Unity
	%          - Potentially requires a lot of calibration
	
	%   - ... <Add more as needed>
	
	\section{Marker-based tracking library} \label{sec:trackinglib}
	% Compares different (AR) tracking libraries.
	
	% Libraries to consider:
		\subsection{Vuforia} \label{ssec:vuforia}
	%   - Vuforia (http://vuforia.com/ and http://developer.vuforia.com/)
	%       - Includes integration with Unity
		\subsection{Unity AR Toolkit (UART)} \label{ssec:uart}
	%   - Unity AR Toolkit (UART) (https://research.cc.gatech.edu/uart/content/contents/)
	%       - Source code and demos hosted on SourceForge (http://sourceforge.net/projects/uart/)
	%       - Seems to be a research project
	%       - Seems easy to use (comes with examples)
	%       - Built for Unity
	%       - Last change to SVN repo was in 2011.
		\subsection{Metaio} \label{ssec:metaio}
	%   - Metaio (http://www.metaio.com/)
	%       - Mainly oriented towards mobile phones, so may not be suitable for this project
	
	%   - ... <Add more as needed>
