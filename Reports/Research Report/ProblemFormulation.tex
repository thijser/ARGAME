\chapter{Problem Formulation} \label{cha:problem}
	% The problem description is taken from BEPSys. It has been shortened 
	% and slightly modified to fit the context of this document.
	While augmented reality research has grown into a mature field over the 
	last years, the aspects of situational awareness and presence of 
	augmented reality (AR) are still quite open research topics. This 
	project is about designing and implementing a collaborative game to 
	explore the different perception of situational awareness, presence and 
	workload in a physical and an AR environment. The game is to be employed 
	as an approximation of collaboratively solving complex problems, as they 
	occur in crime scene investigation when using virtual co-location, i.e. 
	expert remote crime scene investigators to guide local investigators in 
	AR to collaboratively analyse the crime scene. 
	
	It has to be possible to play the game with at least three players: At 
	least two players are present at the same location (physically 
	co-located). At least one player is physically remote but virtually 
	co-located \cite{bepsys}. The game itself should also be unable to be
	completed without either the physically co-located players or the 
	virtually co-located players, however this constraint could be relaxed
	to allow a higher playability of the game (because it would be nice to
	play the game without having to search for a virtually co-located player,
	for example).
	
	%TODO Add our interpretation to avoid ambiguities
