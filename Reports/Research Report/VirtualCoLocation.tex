\section{Virtual Co-location} \label{sec:virtualcolocation}
	% Allowing phsyically remote players to play the game, we need to establish
	% some idea of virtual co-location. This includes the virtualization of the
	% game world as well as the networking functionality required to establish
	% the actual connection.
	Establishing virtual co-location is required to allow physically remote players
	to play the game together. As such, both the virtualization of the game world and
	the networking are considered in virtually co-locating physically remote players.
	Unity has multi player support, because of its master server to handle multi player
	games, but the server could be down at times. There are tutorials on the internet
	to create a basic multi player game that uses the master server to handle requests.
	These tutorials can be used to implement our own multilayer support. Alternatively
	we could provide the players with the means to easily get and exchange their IP
	addresses through other means such as mail.
	Besides the networking we have to look at how we synchronize locations, depending on
	the chosen game we have in order of ascending complexity several options:

	\begin{enumerate}
		\item Use markers with a known locations, this only works if we have a limited size
		      and reasonably fixed playing area which we can prepare ahead of time. This most
					likely comes in the form of a set of markers at the edge and middle of the playing
					field.
		\item Use mobile markers which synchronize between players automatically, for example
		      cards in a card game. This only works if we can trust the play er to keep these markers
					within their screen or if it does not matter if there is no augmentation needed if they
					cannot see a marker.
		\item Object recognition which tracks the locations of objects in the scene,
		      this method only works if there are a number of reasonably stable objects
					within the player's vision.
		\item Combining the output of a compass, a gyroscope and trilocations. This is
		      works regardless of what is visible but requires accurate trilocation which works
					can be quite hard to do without building a heavy rig.
	\end{enumerate}
	Of course a combination of several of the above methods is also possible.

	In the end the goal of all of this is of course to make the remote player feel
	as if he is in the same location as the local player and make the local player feel
	as if the remote player is really with him. So first of all we can look at the
	visualization for the remote player:
	One obvious method might be to put an oculus on the remote player(s) and let
	them see through the eyes of the local player(s), however this is likely to cause nausea.
	Of course we can do away with the oculus but that might result in relative passivism from
	the remote players as they feel they cannot even control their own view while not being
	forced to watch what the other player does.

	We can also let the remote player control one or more avatars within the game world
	and view these through either oculus or screen. This would keep the remote player
	more interested by giving more of a feeling of agency then just viewing through the eyes
	of the other player at the cost of the feeling of connectivity. But this would require
	mapping out a large part of the scene in the virtual world.

	Lastly we can also let the remote player view the world from a birds eye perspective
	this can either be done by mounting a camera above the scene or by rendering it in
	the virtual world, the second case offers increased player agency resulting in
	a better attention to the situation at the cost of having the map the scene fully
	in the virtual world.

	We should also look at how we can make the local player feel as if the remote
	player is with him. One method is to heavily encourage communication. If you
	are talking with someone it is hard to forget their existence. This can be
	encouraged by providing appropriate communication channels such as voice or text
	chat. Another important thing is that the remote player must visibly be doing something
	this can be strengthened by giving him an avatar and by making his actions visibly
	change the world.
