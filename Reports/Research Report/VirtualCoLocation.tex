\section{Virtual Co-location} \label{sec:virtualcolocation}
	% Allowing phsyically remote players to play the game, we need to establish
	% some idea of virtual co-location. This includes the virtualization of the
	% game world as well as the networking functionality required to establish
	% the actual connection.
	Establishing virtual co-location is required to allow physically remote
	players to play the game together. As such, both the virtualization of the
	game world and the networking are considered in virtually co-locating
	physically remote players. Unity has multi player support, because of its
	master server to handle multi player games, but the server could be down at
	times. There are tutorials on the internet to create a basic multi player game
	that uses the master server to handle requests. These tutorials can be used to
	implement our own multilayer support. Alternatively we could provide the
	players with the means to easily get and exchange their IP addresses through
	other means such as mail. Besides the networking we have to look at how we
	synchronize locations, depending on the chosen game we have in order of
	ascending complexity several options:

	\begin{enumerate}
		\item Use markers with known locations. This only works if we have a limited
		      size and reasonably fixed playing area which we can prepare ahead of
					time. This most likely comes in the form of a set of markers on the
					edges and/or center of the playing field.
		\item Use mobile markers which synchronize between players automatically, for
		      example cards in a card game. This only works if we can trust the
					player to keep these markers within their screen or if there is no
					augmentation needed if they cannot see a marker.
		\item Object recognition which tracks the locations of objects in the scene,
		      this method only works if there are a number of reasonably stable
					objects within the player's vision.
		\item Combining the output of a compass, a gyroscope and trilocations. This
		      works regardless of what is visible but requires accurate trilocation
					which works can be quite hard to do without building a heavy rig.
	\end{enumerate}
	Of course a combination of several of the above methods is also possible.

	In the end the goal of the project is of course to make the remote player feel
	as if he is in the same location as the local player and make the local player
	feel as if the remote player is playing together with them. So first of all we can look at the visualization for the remote player:

	One method is to provide an Oculus Rift to the remote player(s) and let them
	see through the eyes of the local player(s). However, this is likely to cause
	nausea. Of course we can display the local player's view on a screen instead,
	but that might result in relative passivism from the remote players, as they
	do not have any control over their own view.

	We can also let the remote player control one or more avatars within the game
	world and view these through either the Oculus Rift or through a screen. This
	would keep the remote player more interested as it would add a greater feeling of immersion than just watching through the eyes of the other player. However, this comes at the cost of the feeling of connectivity. This would also require
	mapping out a large part of the scene in the virtual world.

	Lastly we can also let the remote player view the world from a bird's-eye
	perspective. This can either be done by mounting a camera above the scene or
	by rendering it in the virtual world. The second case offers increased player
	agency resulting in improved situational awareness at the cost of having
	to map the scene fully in the virtual world.

	We should also look at how we can make the local player feel as if the remote
	player is playing together with him. One method is to heavily encourage communication. If you are talking with someone it is hard to forget their existence. This can be encouraged by providing appropriate communication channels such as voice or text chat. Another important thing is that the remote player must visibly be doing something. This can be achieved by giving him an avatar and by making his actions visibly change the world.