\chapter{Proposed Solutions} \label{cha:solution}
	% A description of the proposed solution, including gameplay elements. 
	% The description is taken from the Product Plan. This section also 
	% motivates why the proposed solution is a good solution.

\section{Laser mirror game}
	%TODO: Add references to Problem Analysis chapter where appropriate.
	The goal of the game is to solve a puzzle by controlling laser beams 
	using mirrors in such a way that a predefined target is hit. The game 
	can be played by one or more local players and one or more remote players.
	
	There are cards present for the local players that represent mirror 
	bases. These must be placed on the table, which will be the locations 
	for the mirrors. The local players will be able to see the mirrors they 
	place through the use of AR technology. Each of the local players will 
	only be given a few of the mirror bases needed to solve the puzzle, and 
	as such solving the puzzle requires cooperation from all local players.
	
	The remote players can also see the placed mirrors, and can rotate them 
	to influence the path of the laser beam(s). Only by cooperation between 
	local players (who can only move the mirror bases) and remote players 
	(who can only rotate them) it becomes possible to hit the target and as 
	such solve the puzzle.
	
	The game provides various different types of mirrors with different 
	properties, allowing for more complex puzzles. One example of such a 
	mirror is a colored mirror, and then require the target is hit wi                                                                                                                                                                                                                                                                                                                                                                                                                                                                                                                                                                             th the 
	right (combination of) colors. Another way to make puzzles more complex 
	is requiring that the players combine beams together to create more 
	powerful beams. Other optical components like beam splitters can also be 
	introduced.
	
	The game is designed to stimulate cooperation between the physically 
	co-located players and the physically remote player(s). It does so by 
	dividing abilities required for solving the puzzles amongst all players 
	as follows:
	
	\begin{itemize}
		\item Physically co-located players each get only a part of the 
		      mirror bases required to solve the puzzle, requiring input 
		      from all of these players.
		\item If there are multiple virtually co-located players, each of 
		      these players can only rotate a subset of the mirrors, and 
		      as such input from all virtually co-located players is 
		      required for solving the puzzle. 
		\item Virtually co-located players have the ability to rotate 
		      mirrors while the physically co-located players do not have 
		      this ability, requiring input from both physically as well
		      as virtually co-located players.
	\end{itemize}

	Because of this division of abilities, there is an interdependence 
	between all players (see section \ref{sec:interdependence}), regardless of
	whether these players are physically co-located or not. This replicates the 
	interdependence that exists in the crime scene example as given in the 
	problem formulation (see chapter \ref{cha:problem}).

\section{Platformer}{}
The game starts with the avatar of the remote player appearing
 on the table and a goal appearing above the table. The local
  player has a pile of blocks, each of these blocksexists both
   in the virtual and real world. 
Inside the virtual world a number of pre existing structres and
 obstacles exist making it harder for the remote player to move around. 
The local and remote player must work together in order to get the
 remote player his avatar to the goal. 
Team work is heavily encouraged due to this form of ability 
asymmetry and shared goal.  

\section{FPS Survival game}
Simulair to the platformer the remote players are represented
 as a avatar controlled from the first person perspective and 
 the local player can place blocks. What is different however 
 is that the goal of the game is no longer to reach a given 
 location but to work together in order to protect a given 
 structure from hostile entities. Doing so requires the 
 cooperation of the remote players who can most easily damage 
 the hostile entities and the local players who can hinder the
 movement. Monsters continue to grow stronger until they eventually 
 overwhelm the players. 
The difference between the abilities of a remote player and
a local player give rise to an ability asymmetry and the fact 
that the monsters keep getting stronger means that eventually 
there will also be a difficulty overload forcing extra 
team work. 

\section{Tower defense }
The local players start each with a number of towers each of 
which have certain strengths and abilities. They must place
 these towers on a table in front of them. What they cannot
 but the remote player can see is what paths a number of
 hostile entities will take in order to attack their base.
They must therefor cooperate to place the towers in such 
 positions as to maximize the damage to these hostile 
 entities by communicating the ideal locations of the 
 towers and paths that the hostile entities will take. 
 
 The problem with this idea is that it will very likely create a 
 "puppet master" situation where the remote player takes charge
 of the whole situation and the other players do net get a 
 say in what is going to happen. This could be avoided by a 
 number of extra abilities. This game shows heavy 
 information and ability asymmetry. 