\chapter{Proposed Solution} \label{cha:solution}
	% A description of the proposed solution, including gameplay elements. 
	% The description is taken from the Product Plan. This section also 
	% motivates why the proposed solution is a good solution.
	
	%TODO: Add references to Problem Analysis chapter where appropriate.
	The goal of the game is to solve a puzzle by controlling laser beams 
	using mirrors in such a way that a predefined target is hit. The game 
	can be played by one or more local players and one or more remote players.
	
	There are cards present for the local players that represent mirror 
	bases. These must be placed on the table, which will be the locations 
	for the mirrors. The local players will be able to see the mirrors they 
	place through the use of AR technology. Each of the local players will 
	only be given a few of the mirror bases needed to solve the puzzle, and 
	as such solving the puzzle requires cooperation from all local players.
	
	The remote players can also see the placed mirrors, and can rotate them 
	to influence the path of the laser beam(s). Only by cooperation between 
	local players (who can only move the mirror bases) and remote players 
	(who can only rotate them) it becomes possible to hit the target and as 
	such solve the puzzle.
	
	The game provides various different types of mirrors with different 
	properties, allowing for more complex puzzles. One example of such a 
	mirror is a colored mirror, and then require the target is hit with the 
	right (combination of) colors. Another way to make puzzles more complex 
	is requiring that the players combine beams together to create more 
	powerful beams. Other optical components like beam splitters can also be 
	introduced.
	
	The game is designed to increase situational awareness by requiring the 
	cooperation between the physically co-located players and the physically 
	remote player(s). It does so by dividing abilities required for solving
	the puzzles amongst all players as follows:
	
	\begin{itemize}
		\item Physically co-located players each get only a part of the 
		      mirror bases required to solve the puzzle, requiring input 
		      from all of these players.
		\item If there are multiple virtually co-located players, each of 
		      these players can only rotate a subset of the mirrors, and 
		      as such input from all virtually co-located players is 
		      required for solving the puzzle. 
		\item Virtually co-located players have the ability to rotate 
		      mirrors while the physically co-located players do not have 
		      this ability, requiring input from both physically as well
		      as virtually co-located players.
	\end{itemize}

	Because of this division of abilities, there is an interdependence 
	between all players, regardless of whether these players are physically 
	co-located or not. This replicates the interdependence that exists in 
	the crime scene example as given in the problem formulation (section 
	\ref{cha:problem}).