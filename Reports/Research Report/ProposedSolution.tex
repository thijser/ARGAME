\chapter{Proposed Solutions} \label{cha:solution}
	% A description of the proposed solution, including gameplay elements.
	% The description is taken from the Product Plan. This section also
	% motivates why the proposed solution is a good solution.

\section{Laser mirror game} \label{sec:lasergame}
	The goal of the game is to solve a puzzle by controlling laser beams
	using mirrors in such a way that a predefined target is hit. The game
	can be played by one or more local players and one or more remote players.

	There are cards present for the local players that represent mirror
	bases. These must be placed on the table, which will be the locations
	for the mirrors. The local players will be able to see the mirrors they
	place through the use of AR technology. Each of the local players will
	only be given a few of the mirror bases needed to solve the puzzle, and
	as such solving the puzzle requires cooperation from all local players.

	The remote players can also see the placed mirrors, and can rotate them
	to influence the path of the laser beam(s). Only by cooperation between
	local players (who can only move the mirror bases) and remote players
	(who can only rotate them) it becomes possible to hit the target and as
	such solve the puzzle.

	The game provides various different types of mirrors with different
	properties, allowing for more complex puzzles. One example of such a
	mirror is a colored mirror, and then require the target is hit with the
	right (combination of) colors. Another way to make puzzles more complex
	is requiring that the players combine beams together to create more
	powerful beams. Other optical components like beam splitters can also be introduced.

	The game is designed to stimulate cooperation between the physically
	co-located players and the physically remote player(s). It does so by
	dividing abilities required for solving the puzzles amongst all players
	as follows:

	\begin{itemize}
		\item Physically co-located players each get only a part of the
		      mirror bases required to solve the puzzle, requiring input
		      from all of these players.
		\item If there are multiple virtually co-located players, each of
		      these players can only rotate a subset of the mirrors, and
		      as such input from all virtually co-located players is
		      required for solving the puzzle.
		\item Virtually co-located players have the ability to rotate
		      mirrors while the physically co-located players do not have
		      this ability, requiring input from both physically as well
		      as virtually co-located players.
	\end{itemize}

	Because of this division of abilities, there is an interdependence
	between all players (see section \ref{sec:interdependence}), regardless of
	whether these players are physically co-located or not. This replicates the
	interdependence that exists in the crime scene example as given in the
	problem formulation (see chapter \ref{cha:problem}).

\section{Platformer} \label{sec:platformer}
	The game starts with the avatar of the remote player appearing
	on the table and a goal appearing above the table. The local
  players have a pile of blocks, each of these blocks exists both
  in the virtual and real world.

	Inside the virtual world a number of pre-existing structures and
	obstacles exist making it harder for the remote player to move around.
	The local and remote player must work together in order to get the
	virtual avatar to the goal.

	Team work is heavily encouraged due to the interdependence between the
	players: There is information asymmetry (section \ref{ssec:information})
	because the remote player can see some pre-existing blocks the local
	player cannot see, and the fact that local players can place new blocks
	in the game world provides ability asymmetry (section \ref{ssec:ability}).

\section{FPS Survival game} \label{sec:fpssurvival}
	The goal of the game is to protect a virtual structure from enemies. These
	enemies will come from multiple sides to attack the structure while the
	remote players have to stop them. For doing so, they require the aid of the
	local players, who can place blocks (similar to the platformer game, section
	\ref{sec:platformer}) to block the path of the enemies.

	The local players can block the path of the enemies, but they cannot defeat
	them. The remote players, who view the game from a first-person perspective,
	have to defeat the enemies. Since the perspective of the remote players is
	limited (also partly because of the blocks placed by local players), they
	require information from the local players, who view the scene from above
	and as such have a good view of the entire situation. This creates
	information asymmetry between the players (section \ref{ssec:information}).

	To keep the game challenging, the enemies will grow stronger over time,
	requiring the cooperation between the local and remote players to improve
	as well. Because of the fact that local players can only defeat the enemies,
	and the remote players can block their path, there is a form of ability
	asymmetry between the local and remote players (section \ref{ssec:ability}).

\section{Tower defense} \label{sec:towerdefense}
	The local players start off with a number of towers each
	which have certain strengths and abilities. They must place
	these towers on a table in front of them. What they cannot,
	but the remote player can see is what paths a number of
	hostile entities will take in order to attack their base.
	They must therefore cooperate to place the towers on the right
	positions to be able to defeat the enemies before they escape
	by communicating the ideal locations of the
	towers and paths that the hostile entities will take.

	This game idea is relatively simple, and because of the simplicity,
	complexity can easily be added to make the game more engaging. For
	example, towers could be upgradeable, utilizing resources that
	could be gained either by defeating individual enemies or by defeating
	a wave of enemies. The idea behind these resources is that they are
	shared between players. They could then be used to upgrade
	tower types. As these resources are shared, and they can only be
	used once, players must work together to choose the best upgrade
	available in the given scenario. Enemies should also get procedurally
	stronger because of the increased capabilities of the towers. Resources
	could also be put towards research for new tower types, which players
	could then use.

	Another way to add extra necessary complexity would be through
	introducing an experience system, and to grant towers some experience based
	on what enemy they have defeated. These towers would also over time get
	stronger. Bloons Tower Defense, a game found on the internet developed by Ninja Kiwi, is a tower defense game that uses both resources and experience, to show that these could also be combined. New towers would then be unlocked over time, after completing a certain amount of waves, instead of using resources.

	The problem with this idea is that it will very likely create a
	\emph{puppet master} scenario where the remote player takes charge
	of the whole situation and the other players do net get a
	say in what is going to happen. This could be avoided by a
	number of extra abilities. This game shows heavy
	information and ability asymmetry.

\section{Minesweepers}
	This concept is based on the classic game of minesweeper. There's a grid
	where some of the squares have mines under them. Players each start at a
	random position and have to place flags on locations of mines while avoiding
	mines. The remote player is the only one who can see the numbers around
	squares, so he has to give instructions to the players on the field.

	The difference compared to the classic version of the game is that it's all
	physical. Local players walk around on the field, where they have to take
	careful steps to avoid triggering a mine. This turns the game into some sort
	of Twister variant where people can use special moves to quickly traverse
	the field and find all the mines. Local players co-operate by dividing the
	field into sections that they'll clear in parallel.

	The problem with this concept is the inherent \emph{puppet master} phenomenon.
	Communication between the remote player and local players is very one-sided.
	Local players just receive commands where to step and where to plant flags
	and don't really have any input into the game themselves. Even if that
	problem was solved, the lack of cooperation aspects between local players
	would also be a big problem. Local players don't really have any reason to
	interact with each other. One way to solve that problem would be to require
	multiple players to place a flag.

	There are also some technical challenges with this concept. It requires that
	local players always know their exact position on a large field that they're
	inside of, which would require a lot of markers. Next, the position of their
	legs would have to be determined somehow to ensure that they're not stepping
	on a mine. Finally, a space large enough for a field would likely have to be
	found outside and sunlight doesn't play well with augmented reality devices.
	The game could be transformed into an indoor variant instead, using a board
	and pawns, but this does not solve the one-sided communication problem.

\section{The chosen idea} \label{sec:chosenidea}
	The idea that is chosen is the mirror game idea. The reason for this
	decision is that it is a relatively simple idea, it can create hard
	to solve puzzles, and extra complexity can easily be added. It also
	prevents a \emph{puppet master} scenario by giving both types of players
	about half of the abilities required for solving the game.

	The game offers a strong interdependence between players. This, together
	with the ability asymmetry caused by the separation between moving mirrors
	and rotating them, causes a high need for collaboration between the local
	and remote players. As such, this concept comes closest to the problem
	formulation in creating a collaborative game between physically remote
	players.
