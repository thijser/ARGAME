\chapter{Introduction} \label{cha:introduction}
	% An introduction of what the project is about.
	This chapter introduces the problem in section \ref{sec:problem}, as well as
	the proposed solution in section \ref{sec:solution}. The problem formulation
	as well as the proposed solution present some potential technical challenges 
	to overcome. An analysis of these challeges is given in chapter 
	\ref{cha:analysis}, while details about these challenges are discussed in 
	chapter \ref{cha:technicalchallenges}.

	\section{Problem Formulation} \label{sec:problem}
		% The problem description is taken from BEPSys. It has been shortened 
		% and slightly modified to fit the context of this document.
		While augmented reality research has grown into a mature field over the 
		last years, the aspects of situational awareness and presence of 
		augmented reality (AR) are still quite open research topics. This 
		project is about designing and implementing a collaborative game to 
		explore the different perception of situational awareness, presence and 
		workload in a physical and an AR environment. The game is to be employed 
		as an approximation of collaboratively solving complex problems, as they 
		occur in crime scene investigation when using virtual co-location, i.e. 
		expert remote crime scene investigators to guide local investigators in 
		AR to collaboratively analyse the crime scene. 
		
		It has to be possible to play the game with at least three players: At 
		least two players are present at the same location (physically 
		co-located). At least one player is physically remote but virtually 
		co-located \cite{bepsys}.
	
	\section{Proposed Solution} \label{sec:solution}
		% A description of the proposed solution, including gameplay elements. 
		% The description is taken from the Product Plan. This section also 
		% motivates why the proposed solution is a good solution.
		The goal of the game is to solve a puzzle by controlling laser beams 
		using mirrors in such a way that a predefined target is hit. The game 
		can be played by one or more local players and one or more remote players.
		
		There are cards present for the local players that represent mirror 
		bases. These must be placed on the table, which will be the locations 
		for the mirrors. The local players will be able to see the mirrors they 
		place through the use of AR technology. Each of the local players will 
		only be given a few of the mirror bases needed to solve the puzzle, and 
		as such solving the puzzle requires cooperation from all local players.
		
		The remote players can also see the placed mirrors, and can rotate them 
		to influence the path of the laser beam(s). Only by cooperation between 
		local players (who can only move the mirror bases) and remote players 
		(who can only rotate them) it becomes possible to hit the target and as 
		such solve the puzzle.
		
		The game provides various different types of mirrors with different 
		properties, allowing for more complex puzzles. One example of such a 
		mirror is a colored mirror, and then require the target is hit with the 
		right (combination of) colors. Another way to make puzzles more complex 
		is requiring that the players combine beams together to create more 
		powerful beams. Other optical components like beam splitters can also be 
		introduced.
		
		The game is designed to increase situational awareness by requiring the 
		cooperation between the physically co-located players and the physically 
		remote player(s). It does so by dividing abilities required for solving
		the puzzles amongst all players as follows:
		
		\begin{itemize}
			\item Physically co-located players each get only a part of the 
			      mirror bases required to solve the puzzle, requiring input 
			      from all of these players.
			\item If there are multiple virtually co-located players, each of 
			      these players can only rotate a subset of the mirrors, and 
			      as such input from all virtually co-located players is 
			      required for solving the puzzle. 
			\item Virtually co-located players have the ability to rotate 
			      mirrors while the physically co-located players do not have 
			      this ability, requiring input from both physically as well
			      as virtually co-located players.
		\end{itemize}

		Because of this division of abilities, there is an interdependence 
		between all players, regardless of whether these players are physically 
		co-located or not. This replicates the interdependence that exists in 
		the crime scene example as given in the problem formulation (section 
		\ref{sec:problem}).
