\chapter{Introduction} \label{ch:introduction}
	% An introduction of what the project is about.
	This chapter introduces the problem in section \ref{sec:problem}, as well as
	the proposed solution in section \ref{sec:solution}. The proposed solution
	also presents some potential technical challenges to overcome. Details about 
	these challenges are discussed in chapter \ref{ch:technicalchallenges}.

	\section{Problem Formulation} \label{sec:problem}
		% The problem description is taken from BEPSys. It has been shortened 
		% and slightly modified to fit the context of this document.
		While augmented reality research has grown into a mature field over the 
		last years, the aspects of situational awareness and presence of 
		augmented reality (AR) are still quite open research topics. This 
		project is about designing and implementing a collaborative game to 
		explore the different perception of situational awareness, presence and 
		workload in a physical and an AR environment. The game is to be employed 
		as an approximation of collaboratively solving complex problems, as they 
		occur in crime scene investigation when using virtual co-location, i.e. 
		expert remote crime scene investigators to guide local investigators in 
		AR to collaboratively analyse the crime scene.
		
		It has to be possible to play the game with at least three players: At 
		least two players are present at the same location (physically 
		co-located). At least one player is physically remote but virtually 
		co-located. \cite{bepsys}
	
	\section{Proposed Solution} \label{sec:solution}
		% A description of the proposed solution, including 
		% gameplay elements. This subsection should motivate 
		% why the proposed solution is a good solution.
		
		% Could partly be taken from the Product Plan 
		% once it has been approved.
