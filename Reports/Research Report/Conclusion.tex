\chapter{Conclusion} \label{cha:conclusion}
In short, the mirror game seems the best solution to the problem. It is a viable project
to create in the given time frame, it is a challenging puzzle game even when playing a
different version of it (a single-player version) on your own, and it taxes the communication
of both the remote and the local players, as they both have abilities necessary for
achieving the goal, but they cannot achieve it on their own (although this could be relaxed
to make the game more playable). Also, it is still a challenging project, because of the
technical challenge of integrating AR hardware and recognition of markers with the game
world. Additional complexity, such as colored laser beams/targets, beam splitters, beam mergers
etc. could also be implemented to increase the technical challenge of the problem.

The game and networking will be implemented using Unity, with the Vuforia library
handling augmented reality for the local players. The remote players will likely
be using an Oculus Rift and the local players one of the augmented reality devices
describes in the problem analysis that ends up working best in practice.
	% Summarizes the Research Report and provides
	% an overview and conclusion of the technologies
	% we're going to use.

Using this information, we've built a small demo that uses Vuforia. It places
a laser emitter, mirror and wall on three markers that can be moved around. The
mirror reflects the laser realistically based on the angle of incidence.

\begin{figure}[h]
\includegraphics[width=\textwidth]{demo.png}
\end{figure}
