\chapter{Conclusion} \label{cha:conclusion}
In short, the mirror game seems the best solution to the problem. It is a viable project
to create in the given time frame, it is a challenging puzzle game even when playing a 
different version of it (a single-player version) on your own, and it taxes the communication
of both the remote and the local players, as they both have abilities necessary for
achieving the goal, but they cannot achieve it on their own (although this could be relaxed
to make the game more playable). Also, it is still a challenging project, because of the
technical challenge of integrating AR hardware and recognition of markers with the game
world. Additional complexity, such as colored laser beams/targets, beam splitters, beam mergers
etc. could also be implemented to increase the technical challenge of the problem. For the 
implementation of the game, the networking functionality of Unity will be used to support
multiple players viewing the game world, and the META One as well as the Oculus Rift will
be used in the project, the META One for the local players and the Oculus Rift for the
remote players.
	% Summarizes the Research Report and provides 
	% an overview and conclusion of the technologies 
	% we're going to use.
	% TODO AR tech elaboration.
	
