\chapter{Problem Analysis} \label{cha:analysis}
	% Analysis of the problem description with regards to any potential 
	% difficulties that may arise in the implementation of the proposed 
	% solution. This chapter is a lead-in to the chapter Technical Challenges:
	% that chapter provides an analysis of alternatives and derives a conclusion
	% from that analysis.
	This chapter provides an analysis of the problem description and the 
	proposed solution in terms of technical difficulties that may arise during
	development.
	
	% Because this chapter is closely related to the chapter Technical 
	% Challenges, the sections in that chapter should reflect the difficulties
	% mentioned here.

	% Overview of challenges to consider:
	%   - The use of AR technology presents a challenge on its own
	%       - Solving the AR problem is done using an Oculus Rift and mounted
	%         cameras, in conjunction with an AR library.
	%   - Creating the exprience for the players that they are co-located
	%     even though they are physically remote (virtual co-location)
	%       - We do this by using the Oculus Rift, though that should not be 
	%         mentioned in this chapter
	%   - Creating a set-up that requires physically co-located and physically 
	%     remote players to cooperate (by creating interdependence between them)
	%       - The chapter Technical Challenges may address this by observing 
	%         that this requires interdependance based either on knowledge or on
	%         abilities, but that knowledge is more easily transferred, 
	%         effectively removing the interdependence.
	%   - Since it's a game, playing it should be enjoyable for all players.
	%       - Added this entry because some of our brainstorming ideas missed 
	%         this aspect. Might as well add it in here.
	%
	%   - ... <Add more as needed>
	