\chapter{Problem Analysis} \label{cha:analysis}
	% Includes the challenges surrounding AR and Networking.
	% Also motivates the choices made.
	This chapter provides an overview of the issues and challenges that may
	arise during development of a solution to the problem description. It
	provides an analysis of the problems and possible solutions.

	One of the core challenges of the project is the use of Augmented Reality
	(AR) technology. An analysis of the available options to implement this
	functionality is given in section \ref{sec:ar}. Another important challenge
	is improving situational awareness, which is discussed in section
	\ref{sec:awareness}. The final challenge is the creation of interdependence
	between players in such a way that collaboration from all players is
	required. This challenge is analyzed in section \ref{sec:interdependence}.

	\section{Augmented Reality (AR) Functionality} \label{sec:ar}
		% AR is a core element in the project. As such, we need to compare
		% various AR hardware devices and corresponding ways to implement
		% AR functionality for each device.
		Augmented Reality (AR) is a core aspect of the problem formulation.
		As such, careful analysis has to be done as to how the AR functionality
		can be best implemented to fully address the context of this project.

		We consider three choices for implementing AR functionality: The META
		One (an optical see-through device (\ref{ssec:metaone})), the Oculus
		Rift virtual reality glasses in conjunction with mounted cameras
		(\ref{ssec:oculusrift}) and a smartphone with Google Cardboard
		(\ref{ssec:cardboard}).

		% Hardware devices to consider:
		\subsection{META One} \label{ssec:metaone}
			%   - META One (as indicated by the BEPSys project page)
			%       - Has a limited Field-of-View (around 35 degrees)
			%          - May interfere with the experience of the game.
			%       - Optical see-through glasses means AR works out of the box.

			The META One glasses are optical see-through glasses. \cite{metaone}
			Optical see-through glasses work by projecting a virtual image on
			top of the world you see, effectively implementing a 3D AR
			exprience.

			Because the META One is an optical see-through device that also
			features motion tracking, AR can be implemented simply by
			projecting an image against a black background to the glasses.

			A big drawback of the available META One glasses is their
			Field-of-View, which is 35 degrees. This Field-of-View is way lower
			than the Field-of-View of a person, which may have a negative impact
			on the game experience.

			One of the advantages of this device is that the Software
			Development Kit (SDK) that comes with the glasses has tracking and
			gesture recognition built-in. That allows us to focus on just the
			gameplay aspects.

		\subsection{Oculus Rift} \label{ssec:oculusrift}
			%   - Oculus Rift + mounted cameras (http://oculusvr.com/)
			%       - High Field-of-View (100 degrees)
			%       - VR glasses, so need to project the real world using
			%         cameras.
			%          - Limited resolution creates blurriness.
			%          - Projection can be done from within Unity
			%          - Potentially requires a lot of calibration
			% 		- Using Oculus Rift means we need to implement AR
			%         functionality ourselves (as optical see-through often has
			%         this built-in).
			%
			%         AR Libraries to consider:
			We've built a camera rig for the Oculus Rift that can be used to
			turn it into an augmented reality device. To detect the markers and
			render objects on them in Unity, there are several libraries
			available. Each of these will be discussed in the next sections.

			Oculus offers an SDK for Unity that makes it easy to integrate a
			game with the Rift. \cite{riftdev} The challenge that we'll be
			facing during development is to properly integrate this SDK with the
			augmented reality libraries. Each of the frameworks try to take
			control of the camera in different ways and it's easy to get
			conflicts there. Getting the Rift see-through functionality working
			in Unity on its own and the augmented reality functionality on its
			own is not a challenge.

			The advantage of the Oculus Rift with mounted cameras is that it
			provides complete control over the screen allowing us to completely
			block certain regions of the view or project black objects. It also
			causes the AR content to match up exactly with the real world in
			terms of timing. The field of view is significantly larger as well.

			A disadvantage is the slight latency between head movements and the
			view, which takes work to reduce to a minimum to avoid motion
			sickness and bad hand-eye coordination.

			\subsubsection{Vuforia} \label{sssec:vuforia}
				%   - Vuforia (http://developer.vuforia.com/)
				%       - Includes integration with Unity
				Vuforia is a framework by Qualcomm that allows you to create
				arbitrary markers, import them into Unity and place objects onto
				them. \cite{vuforia} You can then select a webcam and have it
				render the camera images with 3D objects projected onto the
				markers. It's very easy to use and has built-in support for
				virtual reality solutions like GearVR. The tracking quality is
				very good and stable, even with low quality markers (with few
				color transitions).

				Unfortunately it currently only works with the 32-bit version of
				Unity. It also lacks support for the Oculus Rift on the desktop,
				which means that we'll have to build that functionality
				ourselves.

			\subsubsection{Unity AR Toolkit (UART)} \label{sssec:uart}
				%   - Unity AR Toolkit (UART)
				%     (https://research.cc.gatech.edu/uart/content/contents/)
				%       - Source code and demos hosted on SourceForge
				%         (http://sourceforge.net/projects/uart/)
				%       - Seems to be a research project
				%       - Seems easy to use (comes with examples)
				%       - Built for Unity
				%       - Last change to SVN repo was in 2011.
				Unity AR Toolkit \cite{uart} is a project by researchers at the
				Georgia Institute of Technology to develop a set of plugins for
				Unity that make it easy to build augmented reality applications.

				Unfortunately the project was abandoned in 2011 and the tracking
				library it's based on (ARToolKitPlus) was abandoned in 2006. The
				documentation is also severely lacking. The camera also needs to
				be calibrated to work with it. This project also only works with
				the 32-bit version of Unity and lacks native Rift support.

				The one advantage of this library is that it moves the 3D
				objects by default, instead of the camera. That may make it
				easier to reconstruct the world with its mirrors in Unity.

			\subsubsection{Metaio} \label{sssec:metaio}
				%   - Metaio (http://www.metaio.com/)
				%       - Mainly oriented towards mobile phones, so may not be
				%         suitable for this project
				Metaio is a well-supported project, much like Vuforia. The
				difference is that it has native support for see-through stereo
				glasses, which makes it well suitable for both the META One and
				Oculus Rift solution. \cite{metaio}

				Unlike any of the other solutions, this framework works
				natively on Unity 64-bit. It does have the limit of only working
				with OpenGL, but we don't expect that to be a problem.

		%   - ... <Add more as needed>
		\subsection{Google Cardboard} \label{ssec:cardboard}
			Google Cardboard can be used to implement the same concept for
			augmented reality as the Oculus Rift. It allows you to turn a
			smartphone into a VR headset by mounting lenses in front of it. This
			can then be used to implement AR by overlaying 3D objects over the
			phone camera image. \cite{cardboard}

			The advantage of this approach over the Oculus Rift is that it's
			much more convenient. The solution would be completely mobile,
			meaning that there are no cables getting in the way (a problem we
			ran into during testing). It's also much cheaper, making it easier
			for local players to join by simply owning a smartphone. The
			disadvantage is that there would be no depth, since most smartphones
			don't have a stereo camera.

			The libraries mentioned in the previous section are all suitable for
			mobile usage. In fact, they were designed for it and just happen to
			allow for desktop usage as well. This currently seems like the most
			promising alternative, but it will require further testing.

		\subsection{Markers} \label{ssec:markers}
			An essential component of augmented reality systems are markers.
			Markers are physical objects that are used by the augmented reality
			framework to establish the position of the player and objects in the
			virtual world (tracking). There are two main types of tracking:

			\begin{itemize}
				\item \textbf{Marker tracking:} Special asymmetric patterns
					similar to QR codes are printed to cards. These are designed
					to be efficiently detected by a computer

				\item \textbf{Markerless tracking:} Tracking with arbitrary
					images as opposed to specially designed markers, therefore
					known as markerless tracking. This name is a bit misleading,
					because these images are still typically chosen based on
					properties like contrasting colors and sharp edges that are
					easier to detect.
			\end{itemize}

			The advantage of markerless tracking is that they look more
			aesthetically pleasing to the players, but the performance may be
			worse than specially designed patterns. For that reason, we're going
			to use a hybrid form that combines an easy to scan pattern with a
			nice looking image of the object it represents (like a mirror). An
			example of this concept are graphical QR codes. \cite{gqr}

	\section{Situational Awareness} \label{sec:awareness}
		% Improving situational awareness is part of the main goal of the
		% project. We should indicate the steps needed to achieve this, which
		% can be based on (possibly) a large range of scientific articles.
		This project is about exploring the different perception of situational
		awareness, presence and workload in a physical and an AR environment
		(see chapter \ref{cha:problem}). As such, situational awareness plays a
		key role in this project.

		Before considering exactly how situational awareness plays a role in this
		project, it is important to define precisely what situational awareness
		means. According to \cite{endsley}, situational awareness is defined as
		"the perception of the elements in the environment within a volume of
		time and space, comprehension of their meaning, and the projection of
		their status in the near future". In other words, situational awareness
		means to fully understand the situation, and be able to predict what is
		going to happen next. This also includes understanding any risks the
		situation brings.

		Now that the concept of situational awareness has been defined, the
		importance of situational awareness for this project needs to be
		considered. This project aims to approximate a situation in which the
		players need to collaboratively solve complex problems, as they occur in
		crime scene investigations (see chapter \ref{cha:problem}). On a crime
		scene, the investigator needs to analyse and understand the situation,
		so there is a need for situational awareness in the problem description.
		The game will need to replicate a similar need for situational awareness
		to be an accurate approximation of the context of this project.

	\section{Interdependence between players} \label{sec:interdependence}
		% Creating interdependence between players requires them to work
		% together. This can be done in several ways. We need to elaborate on
		% the various ways in which this can be achieved.
		The problem formulation states that the game is to be employed as an
		approximation of collaboratively solving complex problems. In order to
		motivate players of the game to collaborate, there is a need to create
		a form of interdependence amongst the players, as mentioned in
		\cite{zagal}. One way to do this is to create an asymmetry of abilities.
		Other ways to create interdependence are explained in the following
		subsections.
		\subsection{Asymmetry of abilities} \label{ssec:ability}
			The main reason to co-operate is the asymmetry of abilities between
			the players involved. For example: physically co-located players can
			alter the game world, while virtually co-located players can guide
			characters to a certain goal utilizing the altered game environment.
			One thing to note is that a "puppet master" scenario should be
			avoided. This scenario happens when one player can do everything
			except for a few required tasks, and uses the other players to
			execute these tasks. In this case, the other players will have less
			involvement with the shared goal, and the amount of co-operation
			will go down.
		\subsection{Asymmetry of information} \label{ssec:information}
			Asymmetry of information could be used as another reason for the
			players to co-operate. It means that both types of players (both the
			physically co-located and the virtually co-located) have different,
			separate, parts of the information required to complete the game. In
			this case, a "puppet master" scenario should also be avoided. Such a
			scenario can occur here when one type of player has enough
			information to infer nearly all information.

		\subsection{Information overload} \label{ssec:overload}
			Another reason for players to co-operate could be information
			overload. This means that, while the game could technically be
			completed by a single player, the amount of incoming information is
			too large for one player to handle. An example would be in Call of
			Duty, where people work together because there are too many enemies
			that walk around and shoot at them. A single player could
			potentially beat the game by themselves, but this is not really
			feasible considering the amount of enemies and the amount of
			information coming in continuously.

	\subsection{Virtual Co-location} \label{sec:virtualcolocation}
	% Allowing phsyically remote players to play the game, we need to establish
	% some idea of virtual co-location. This includes the virtualization of the
	% game world as well as the networking functionality required to establish
	% the actual connection.
	Establishing virtual co-location is required to allow physically remote
	players to play the game together. As such, both the virtualization of the
	game world and the networking are considered in virtually co-locating
	physically remote players. Unity has multi player support, because of its
	master server to handle multi player games, but the server could be down at
	times. There are tutorials on the internet to create a basic multi player game
	that uses the master server to handle requests. These tutorials can be used to
	implement our own multilayer support. Alternatively we could provide the
	players with the means to easily get and exchange their IP addresses through
	other means such as mail. Besides the networking we have to look at how we
	synchronize locations, depending on the chosen game we have in order of
	ascending complexity several options:

	\begin{enumerate}
		\item Use markers with known locations. This only works if we have a limited
		      size and reasonably fixed playing area which we can prepare ahead of
					time. This most likely comes in the form of a set of markers on the
					edges and/or center of the playing field.
		\item Use mobile markers which synchronize between players automatically, for
		      example cards in a card game. This only works if we can trust the
					player to keep these markers within their screen or if there is no
					augmentation needed if they cannot see a marker.
		\item Object recognition which tracks the locations of objects in the scene,
		      this method only works if there are a number of reasonably stable
					objects within the player's vision.
		\item Combining the output of a compass, a gyroscope and trilocations. This
		      works regardless of what is visible but requires accurate trilocation
					which works can be quite hard to do without building a heavy rig.
	\end{enumerate}
	Of course a combination of several of the above methods is also possible.

	In the end the goal of the project is of course to make the remote player feel
	as if he is in the same location as the local player and make the local player
	feel as if the remote player is playing together with them. So first of all we can look at the visualization for the remote player:

	One method is to provide an Oculus Rift to the remote player(s) and let them
	see through the eyes of the local player(s). However, this is likely to cause
	nausea. Of course we can display the local player's view on a screen instead,
	but that might result in relative passivism from the remote players, as they
	do not have any control over their own view.

	We can also let the remote player control one or more avatars within the game
	world and view these through either the Oculus Rift or through a screen. This
	would keep the remote player more interested as it would add a greater feeling of immersion than just watching through the eyes of the other player. However, this comes at the cost of the feeling of connectivity. This would also require
	mapping out a large part of the scene in the virtual world.

	Lastly we can also let the remote player view the world from a bird's-eye
	perspective. This can either be done by mounting a camera above the scene or
	by rendering it in the virtual world. The second case offers increased player
	agency resulting in improved situational awareness at the cost of having
	to map the scene fully in the virtual world.

	We should also look at how we can make the local player feel as if the remote
	player is playing together with him. One method is to heavily encourage communication. If you are talking with someone it is hard to forget their existence. This can be encouraged by providing appropriate communication channels such as voice or text chat. Another important thing is that the remote player must visibly be doing something. This can be achieved by giving him an avatar and by making his actions visibly change the world.
