\chapter{Problem Analysis} \label{cha:analysis}
	% Includes the challenges surrounding AR and Networking.
	% Also motivates the choices made.
	This chapter provides an analysis of the problem description of problems
	and challenges that may arise during development. It provides an analysis 
	of the problems and possible solutions that can be used in order to 
	solve these problems. 
	
	\section{Augmented Reality (AR) Functionality} \label{sec:ar}
		% AR is a core element in the project. As such, we need to compare
		% various AR hardware devices and corresponding ways to implement
		% AR functionality for each device.
	
		% Hardware devices to consider:
		\subsection{META One} \label{ssec:metaone}
			%   - META One (as indicated by the BEPSys project page)
			%       - Has a limited Field-of-View (around 35 degrees)
			%          - May interfere with the experience of the game.
			%       - Optical see-through glasses means AR works out of the box.
		\subsection{Oculus Rift} \label{ssec:oculusrift}
			%   - Oculus Rift + mounted cameras (http://oculusvr.com/)
			%       - High Field-of-View (100 degrees)
			%       - VR glasses, so need to project the real world using cameras.
			%          - Limited resolution creates blurriness.
			%          - Projection can be done from within Unity
			%          - Potentially requires a lot of calibration
			% 		- Using Oculus Rift means we need to implement AR functionality 
			%         ourselves (as optical see-through often has this built-in).
			%         AR Libraries to consider:
			\subsubsection{Vuforia} \label{sssec:vuforia}
				%   - Vuforia (http://vuforia.com/ and http://developer.vuforia.com/)
				%       - Includes integration with Unity
			\subsubsection{Unity AR Toolkit (UART)} \label{sssec:uart}
				%   - Unity AR Toolkit (UART) (https://research.cc.gatech.edu/uart/content/contents/)
				%       - Source code and demos hosted on SourceForge (http://sourceforge.net/projects/uart/)
				%       - Seems to be a research project
				%       - Seems easy to use (comes with examples)
				%       - Built for Unity
				%       - Last change to SVN repo was in 2011.
			\subsubsection{Metaio} \label{sssec:metaio}
				%   - Metaio (http://www.metaio.com/)
				%       - Mainly oriented towards mobile phones, so may not be suitable for this project
	
		%   - ... <Add more as needed>
	
	\section{Situational Awareness} \label{sec:awareness}
		% Improving situational awareness is part of the main goal of the project.
		% We should indicate the steps needed to achieve this, which can be based 
		% on (possibly) a large range of scientific articles.
	
	\section{Interdependence between players} \label{sec:interdependence}
		% Creating interdependence between players requires them to work together.
		% This can be done in several ways. We need to elaborate on the various ways
		% in which this can be achieved.
		
	\section{Analysis Conclusions} \label{sec:analysisconclusion}
		% Provides the coices we made for the abovementioned problems along with a short
		% motivation based on the above analysis.
