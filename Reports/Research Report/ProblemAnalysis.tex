\chapter{Problem Analysis} \label{cha:analysis}
	% Includes the challenges surrounding AR and Networking.
	% Also motivates the choices made.
	This chapter provides an analysis of the problem description of problems
	and challenges that may arise during development. It provides an analysis 
	of the problems and possible solutions that can be used in order to 
	solve these problems. 
	
	One of the core challenges of the project is the use of Augmented Reality 
	(AR) technology. An analysis of the available options to implement this 
	functionality is given in section \ref{sec:ar}. Another important challenge
	is improving situational awareness, which is discussed in section 
	\ref{sec:awareness}. The last challenge is the creation of interdependence 
	between players in such a way that requires collaboration from all players.
	This challenge is analyzed in section \ref{sec:interdependence}. 
	
	Lastly, a conclusion based on the analyses of these challenges is given in
	section \ref{sec:analysisconclusion}.
	
	\section{Augmented Reality (AR) Functionality} \label{sec:ar}
		% AR is a core element in the project. As such, we need to compare
		% various AR hardware devices and corresponding ways to implement
		% AR functionality for each device.
		Augmented Reality (AR) is a core aspect of the problem formulation. 
		As such, careful analysis has to be done as to how the AR functionality
		can be best implemented to fully address the context of this project.
		
		We consider two choices for implementing AR functionality: The META One,
		an optical see-through device (\ref{ssec:metaone}), and the Oculus Rift 
		Virtual Reality glasses in conjunction with mounted cameras 
		(\ref{ssec:oculusrift}).
	
		% Hardware devices to consider:
		\subsection{META One} \label{ssec:metaone}
			%   - META One (as indicated by the BEPSys project page)
			%       - Has a limited Field-of-View (around 35 degrees)
			%          - May interfere with the experience of the game.
			%       - Optical see-through glasses means AR works out of the box.
			
			The META One glasses are optical see-through glasses. Optical 
			see-through glasses work by projecting a virtual image on top of the
			world you see, effectively implementing a 3D AR exprience.
			
			Because the META One is an optical see-through device that also 
			features motion tracking, AR can be implemented simply by 
			projecting an image against a black background to the glasses.
			
			A big drawback of the available META One glasses is their 
			Field-of-View, which is 35 degrees. This Field-of-View is way lower
			than the Field-of-View of a person, which may have a negative impact 
			on the game experience.

			One of the advantages of this device is that the SDK has tracking
			and gesture recognition built-in. That allows us to focus on just
			the gameplay aspects.
			
		\subsection{Oculus Rift} \label{ssec:oculusrift}
			%   - Oculus Rift + mounted cameras (http://oculusvr.com/)
			%       - High Field-of-View (100 degrees)
			%       - VR glasses, so need to project the real world using cameras.
			%          - Limited resolution creates blurriness.
			%          - Projection can be done from within Unity
			%          - Potentially requires a lot of calibration
			% 		- Using Oculus Rift means we need to implement AR functionality 
			%         ourselves (as optical see-through often has this built-in).
			%         AR Libraries to consider:
			We've built a camera rig for the Oculus Rift that can be used to
			turn it into an augmented reality device. To detect the markers and
			render objects on them in Unity, there are several libraries
			available. Each of these will be discussed in the next sections.

			Oculus offers an SDK for Unity that makes it easy to integrate a
			game with the Rift. The challenge that we'll be facing during
			development is to properly integrate this SDK with the augmented
			reality libraries. Each of the frameworks try to take control of the
			camera in different ways and it's easy to get conflicts there.
			Getting the Rift see-through functionality working in Unity on its
			own and the augmented reality functionality on its own is not a
			challenge.

			\subsubsection{Vuforia} \label{sssec:vuforia}
				%   - Vuforia (http://vuforia.com/ and http://developer.vuforia.com/)
				%       - Includes integration with Unity
				Vuforia is a framework by Qualcomm that allows you to create
				arbitrary markers, import them into Unity and place objects onto
				them. You can then select a webcam and have it render the camera
				images with 3D objects projected onto the markers. It's very
				easy to use and has built-in support for virtual reality
				solutions like GearVR. The tracking quality is very good and
				stable, even with low quality markers (with few color transitions).

				Unfortunately it currently only works with the 32-bit version of
				Unity. It also lacks support for the Oculus Rift on the desktop,
				which means that we'll have to build that functionality ourselves.

			\subsubsection{Unity AR Toolkit (UART)} \label{sssec:uart}
				%   - Unity AR Toolkit (UART) (https://research.cc.gatech.edu/uart/content/contents/)
				%       - Source code and demos hosted on SourceForge (http://sourceforge.net/projects/uart/)
				%       - Seems to be a research project
				%       - Seems easy to use (comes with examples)
				%       - Built for Unity
				%       - Last change to SVN repo was in 2011.


			\subsubsection{Metaio} \label{sssec:metaio}
				%   - Metaio (http://www.metaio.com/)
				%       - Mainly oriented towards mobile phones, so may not be suitable for this project
	
		%   - ... <Add more as needed>
	
	\section{Situational Awareness} \label{sec:awareness}
		% Improving situational awareness is part of the main goal of the project.
		% We should indicate the steps needed to achieve this, which can be based 
		% on (possibly) a large range of scientific articles.
		This project is about exploring the different perception of situational 
		awareness, presence and workload in a physical and an AR environment 
		(see chapter \ref{cha:problem}). As such, situational awareness plays a 
		key role in this project.
		
	
	\section{Interdependence between players} \label{sec:interdependence}
		% Creating interdependence between players requires them to work together.
		% This can be done in several ways. We need to elaborate on the various ways
		% in which this can be achieved.
		The problem formulation states that the game is to be employed as an 
		approximation of collaboratively solving complex problems. In order to
		motivate players of the game to collaborate, there is a need to create
		a form of interdependence amongst the players. One way to do this is to 
		create an asymmetry between either the information that the players have or
		an asymmetry of abilities, as explained in the following subsection. 
		\subsection{Reasons to co-operate}
			The main reason to co-operate is the asymmetry of abilities between
			the players involved. For example: the physically co-located players
			have the ability to put down mirrors, and the virtually co-located
			players can then rotate the mirrors. Furthermore, information
			asymmetry can also be implemented, for example by allowing only the
			virtually co-located players to see the obstacles. Because of physical
			separation, asymmetry should be focused on physical abilities.
		
	\section{Virtual Co-location} \label{sec:virtualcolocation}
		Establishing virtual co-location is required to allow physically remote players
		to play the game together. As such, both the virtualization of the game world and
		the networking are considered in virtually co-locating physically remote players.
		Unity has multiplayer support, because of its master server to handle multiplayer
		games, but the server could be down at times. There are tutorials on the internet
		to create a basic multiplayer game that uses the master server to handle requests.
		These tutorials can be used to implement our own multiplayer support.
		% Allowing phsyically remote players to play the game, we need to establish
		% some idea of virtual co-location. This includes the virtualization of the 
		% game world as well as the networking functionality required to establish
		% the actual connection.
		
	\section{Analysis Conclusions} \label{sec:analysisconclusion}
		% Provides the coices we made for the abovementioned problems along with a short
		% motivation based on the above analysis.
