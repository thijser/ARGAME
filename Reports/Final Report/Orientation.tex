\chapter{Orientation} \label{cha:orientation}

  This chapter provides an overview of the orientation phase of the project.
  It shows an analysis of the project requirements, and the decisions that
  have been made during the project regarding choices of frameworks and
  libraries as well as gameplay elements.

  For a more in-depth view on the research that has been done leading to
  these decisions, please refer to the Research Report in appendix
  \ref{app:researchreport}.

  \section{Project Description} \label{sec:projectdescription}
    % TODO: Write description of the project.
    %       This can be similar to the problem formulation in the Research
    %       Report.

  \section{Final Product} \label{sec:finalproduct}
    % TODO: Write description of the final product.

  \section{Software Design Methods} \label{sec:designmethods}
    This section describes the design methods that were used during the
    project. It illustrates the methods that were used to develop and
    coordinate the project during the development phase.

    \subsection{Design Process} \label{ssec:designprocess}
      In designing and implementing the product, it is important that
      requirements can be changed quickly and without much problems. This is
      not because the requirements are likely to change from the client
      side, but because the choice of AR technology may change over the
      course of the project because of technical issues. The available Virtual
      and Augmented Reality glasses are mostly still in development, and as
      such this may affect the technical viability of each device.

      To deal with such changes, we use the Scrum methodology. The Scrum
      methodology describes a set of rules that, amongst others, makes it
      easier to deal with various changes during the development process.
      For a complete description of the rules that the Scrum methodology
      describes, please refer to the Scrum guide, available at
      \url{http://www.scrumguides.org/scrum-guide.html}.

    \subsection{Organization} \label{ssec:organization}
      To be able to simulaneously work on the project without conflicts, we
      use Git as a version control system. The project is stored remotely on
      Github, ensuring the work is efficiently shared between all team members.

      To coordinate and divide the tasks, as well as to maintain the items in
      the Scrum backlog, we use Trello. Trello is an online service that
      provides a dynamic way to organize items in various lists. We created
      lists to keep track of which items were in the backlog, which items were
      being worked on and which items were already done.

      The project is licensed under the terms of the MIT license. We chose
      this license because it allows other developers to learn from this
      project. Additionally, since this project is done as a part of a
      research project, we believe making this project open source may
      help future researchers in the same field. The full terms of the MIT
      license can be found at \url{http://opensource.org/licenses/mit-license.html}

    \subsection{Design Architecture} \label{ssec:designarchitecture}
      Because the product is a game and the goal of the project is more
      focused on the game mechanics rather than the underlying engine, we chose
      to use Unity as a starting point. Unity provides a platform-independant
      IDE for developing games, and offers many features commonly used in
      games.

      Using Unity means that the project architecture is bound to the
      loosely coupled component-based architecture that Unity provides,
      although it is possible to include principles from object-oriented
      programming to some extent.
