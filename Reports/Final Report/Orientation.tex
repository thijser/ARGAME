\chapter{Orientation} \label{cha:orientation}
\pagenumbering{arabic}
  This chapter provides an overview of the orientation phase of the project.
  It shows an analysis of the project requirements, and the decisions that
  have been made during the project regarding choices of frameworks and
  libraries as well as gameplay elements. It also functions as a thorough
  introduction to this report.

  For a more in-depth view on the research that has been done leading to
  these decisions, please refer to the Research Report in appendix
  \ref{app:researchreport}.

  \section{Project Description} \label{sec:projectdescription}
    While augmented reality research has grown into a mature field over the
    last years, the aspects of situational awareness and presence of
    augmented reality (AR) are still quite open research topics. This
    project is about designing and implementing a collaborative game to
    explore the different perception of situational awareness, presence and
    workload in a physical and an AR environment. The game is to be employed
    as an approximation of collaboratively solving complex problems, as they
    occur in crime scene investigation when using virtual co-location, i.e.
    expert remote crime scene investigators to guide local investigators in
    AR to collaboratively analyse the crime scene.

  \section{Final Product} \label{sec:finalproduct}
    The goal of the game is to solve a puzzle by controlling laser beams
    using mirrors in such a way that a predefined target is hit. The game
    can be played by one or more local players and one or more remote players.

    There are cards present for the local players that represent mirror
    bases. These must be placed on the table, which will be the locations
    for the mirrors. The local players will be able to see the mirrors they
    place through the use of AR technology. Each of the local players will
    only be given a few of the mirror bases needed to solve the puzzle, and
    as such solving the puzzle requires cooperation from all local players.

    The remote players can also see the placed mirrors, and can rotate them
    to influence the path of the laser beam(s). Only by cooperation between
    local players (who can only move the mirror bases) and remote players
    (who can only rotate them) it becomes possible to hit the target and as
    such solve the puzzle.

    The game provides various different types of mirrors with different
    properties, allowing for more complex puzzles. One example of such a
    mirror is a colored mirror, and then require the target is hit with the
    right (combination of) colors. Another way to make puzzles more complex
    is requiring that the players combine beams together to create more
    powerful beams. Other optical components like beam splitters can also be
    introduced.

    The game is designed to stimulate cooperation between the physically
    co-located players and the physically remote player(s). It does so by
    dividing abilities required for solving the puzzles amongst all players
    as follows:

    \begin{itemize}
      \item Physically co-located players each get only a part of the
            mirror bases required to solve the puzzle, requiring input
            from all of these players.
      \item If there are multiple physically remote players, each of
            these players can only rotate a subset of the mirrors, and
            as such input from all physically remote players is
            required for solving the puzzle.
      \item Physically remote players have the ability to rotate
            mirrors while the physically co-located players do not have
            this ability, requiring input from both physically co-located
            as well as physically remote players.
    \end{itemize}

  \section{Software Design Methods} \label{sec:designmethods}
    This section describes the design methods that were used during the
    project. It illustrates the methods that were used to develop and
    coordinate the project during the development phase.

    \subsection{Design Process} \label{ssec:designprocess}
		In designing and implementing the product, it is important that
		requirements can be changed quickly and without much problems. This is
		not because the requirements are likely to change from the client
		side, but because the choice of AR technology may change over the
		course of the project because of technical issues. The available Virtual
		and Augmented Reality glasses are mostly still in development, and as
		such this may affect the technical viability of each device.
		
		To deal with such changes, we use the Scrum methodology. The Scrum
		methodology describes a set of rules that, amongst others, makes it
		easier to deal with various changes during the development process.
		For a complete description of the rules that the Scrum methodology
		describes, please refer to the Scrum guide, available at
		\url{http://www.scrumguides.org/scrum-guide.html}. The methodology
		in graph form can be seen in \ref{fig:scrum}. The methodology
		shown is nearly identical to the methodology we use, the only 
		exception being that our sprints only last one week instead of two.
		\begin{figure}
			\centering
			\includegraphics[width=\textwidth]{Scrum}
			\caption{Scrum.}
			\label{fig:scrum}
		\end{figure}

    \subsection{Organization} \label{ssec:organization}
      To be able to simulaneously work on the project without conflicts, we
      use Git as a version control system. The project is stored remotely on
      GitHub, ensuring the work is efficiently shared between all team members.
      Also, Git stores all commits that have been done. These can be reviewed
      on GitHub, and it is possible to go back to a commit stage if we
      absolutely need to, in case of something horribly wrong happening to
      the project.

      To coordinate and divide the tasks, as well as to maintain the items in
      the Scrum backlog, we use Trello. Trello is an online service that
      provides a dynamic way to organize items in various lists. It does
      so by using cards as bullet points in a list. It is also possible to
      assign certain people to those cards, which can be used to visibly
      divide tasks among the group members. Using labels, each card can also
      be categorized as a task relating to a certain part or parts of the
      project, such as software engineering, graphics design, networking,
      etc. We created lists to keep track of which items were in the backlog, 
      which items were being worked on and which items were already done.
      This allowed us to easily see what was being done and what was done.

      The project is licensed under the terms of the MIT license. We chose
      this license because it allows other developers to learn from this
      project. Additionally, since this project is done as a part of a
      research project, we believe making this project open source may
      help future researchers in the same field. The full terms of the MIT
      license can be found at \url{http://opensource.org/licenses/mit-license.html}

	  %TODO Move this paragraph to correct sections in the Quality Assurance chapter.
      To ensure the C\# source code in the project meets common coding standards
      (as set by Microsoft), we use the code analysis tools FxCop and StyleCop
      in combination with SonarQube. FxCop performs static code analysis, like
      code complexity and some naming conventions. StyleCop, on the other hand,
      focuses more on code style which includes use of spacing and
      documentation as well as other factors. SonarQube is a platform that
      unifies the reports from these tools and provides a clean overview of the
      combined issues found by FxCop and StyleCop, as well as some simple metrics
      SonarQube has built-in.

    \subsection{Design Architecture} \label{ssec:designarchitecture}
      Because the product is a game and the goal of the project is more
      focused on the game mechanics rather than the underlying engine, we chose
      to use Unity as a starting point. Unity provides a platform-independant
      IDE for developing games, and offers many features commonly used in
      games.

      Using Unity means that the project architecture is bound to the
      loosely coupled component-based architecture that Unity provides,
      although it is possible to include principles from object-oriented
      programming to some extent.
      
    \section{Process}\label{sec:process}
      This section describes the process of the project. Overall, the project
      was like a huge sinusoidal wave of "It's working" and "It's not working",
      which oscillated between those two points several times a day.
      
      \subsection{Pre-project preparations} \label{ssec:preparations}
      Before the project started, we had a meeting halfway through March
      with our coach about what the project entails and what is currently
      possible, given the hardware that we have today. After a brainstorm
      session and a pitch session with our coach and client, we were shown 
      what kind of hardware the TU Delft has available, and also what the 
      limitations of this hardware are.
      
      \subsection{Research} \label{ssec:research}
      The project started halfway through April, at the beginning of Q4.
      This is the "in schedule" time for the project. The first meeting
      started with a definite brainstorm session about what kind of game
      should be created. The rest of the week was our research phase, including
      topics like what hardware would be required, what functionality of Unity 
      would be needed, as well as other potential issues. Also, the rest of the 
      week was dedicated to testing out various parts of the required software 
      and hardware, as well as defining the requirements according to the MoSCoW 
      model in the product plan, shown in the product plan in Appendix \ref{app:productplan}
      
      \subsection{Research report} \label{ssec:researchreport}
      The first two weeks were the main research phase. This entailed that the
      research report had to be written. The second week was partly devoted
      to writing the report, and also to testing out more functionality of
      the software and hardware. The first game object models were also
      developed during this time, and there were plans for a first demo at the end
      of week 3 or at the beginning of week 4.
      
      \subsection{Programming the basic game} \label{ssec:basics}
      The next two weeks revolved around creating game objects and gameplay.
      After some playing around with markers and AR glasses, as well as
      creating the basic and advanced game objects, a first demo was also
      developed during this time. This phase also saw unit testing of game
      elements, as well as heavy usage of StyleCop and SonarQube to clean up
      resp. refactor code in order to deliver clean code to SIG, for our first
      submission.
      
      \subsection{Demo + the OpenCV server} \label{ssec:firstdemo}
      Week 5 finally saw the first demo being demonstrated to the coach. The
      coach was satisfied, but there was a lot to be done before the project
      could be considered finished. During this week, networking was revamped
      and development of a server that makes use of the computer vision library
      OpenCV started. This decision was based on various technical challenges we 
      came across during development. The reason for this decision can be found 
      in the Implementation chapter (chapter \ref{cha:implementation}).
      
      \subsection{Restructuring the entire project}
      Week 6 began with a massive restructuring of the project. Right before
      the code was to be handed in to SIG, Unity failed to build the project
      completely. There were no errors in the scripts, but errors in the build
      came directly from the compiler. This caused us to move everything that
      we wanted to move from the original project to the new project. After
      completing this task, development could continue on the OpenCV server.
      
      \subsection{The new projection code}
      Week 7 began with further development on the OpenCV server and the code
      needed for correct projection of the META One. The projection code base
      was overhauled on Monday, mainly to allow for better unit testing. During
      this week, the midterm meeting also took place, and we could show off
      what we had up until that point. The coach and the customer(s) (Stephan
      could not be there, so he sent some of his colleagues to check in on
      how the project went) were impressed, but they also said that development
      had to continue for the time being. At the end of week 7, the server was
      completed, but now the true challenge began: integrating all the software
      into one single product.
      
      Week 8 started with the integration of all the parts of the software.
      From the start on, this proved quite challenging. Players could rotate
      mirrors by rotating the markers (which shouldn't happen, as this would
      kill off the co-operative element of the game), as well as other things.
      As such, development on projection code was once again necessary.
      Throughout this week, massive progress was made towards integration of 
      the various parts of the software, especially from early Wednesday on.
      Because of the progress considering integration of software, level design
      could finally start.
      
      \subsection{Further integration}
      Week 9 was even further devoted to integration of all the various bits and
      pieces of the software. 
      
      \subsection{The process table} \label{ssec:processtable}
      The following table gives a nice and short overview regarding what was done
      every week. An X in a particular subproject and week means that, during that
      week, that subproject was developed further.\\
      \begin{table}[!ht]
	      \begin{tabular}{| c | c | c | c | c | c | c | c | c | c | c |}
	      	      \hline
	      	      Weeks & 1 & 2 & 3 & 4 & 5 & 6 & 7 & 8 & 9 & 10 \\
	      	      \hline
	      	      Research & X & X & \space & \space & \space & \space & \space & \space & \space & \space \\
	      	      \hline
	      	      Final report & \space & \space & \space & X & X & X & X & X & \space & \space \\
	      	      \hline
	      	      Networking & X & X & \space & \space & X & X & X & \space & \space & \space \\
	      	      \hline
	      	      Augmented reality & \space & \space & \space & X & X & X & X & X & \space & \space \\
	      	      \hline
	      	      Gameplay & X & X & X & X & \space & \space & \space & \space & \space & \space \\
	      	      \hline
	      	      Testing & \space & \space & X & X & X & X & X & X & \space & \space \\
	      	      \hline
	      	      Projection & \space & \space & \space & X & X & X & X & X & \space & \space \\
		      	  \hline
		      	  Level design & \space & \space & \space & \space & \space & \space & \space & X & \space & \space \\
		      	  \hline
	      \end{tabular}
      \end{table}
