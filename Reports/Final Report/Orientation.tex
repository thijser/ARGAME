\chapter{Orientation} \label{cha:orientation}

  This chapter provides an overview of the orientation phase of the project.
  It shows an analysis of the project requirements, and the decisions that
  have been made during the project regarding choices of frameworks and
  libraries as well as gameplay elements.

  For a more in-depth view on the research that has been done leading to
  these decisions, please refer to the Research Report in appendix
  \ref{app:researchreport}.

  \section{Project Description} \label{sec:projectdescription}
    While augmented reality research has grown into a mature field over the
    last years, the aspects of situational awareness and presence of
    augmented reality (AR) are still quite open research topics. This
    project is about designing and implementing a collaborative game to
    explore the different perception of situational awareness, presence and
    workload in a physical and an AR environment. The game is to be employed
    as an approximation of collaboratively solving complex problems, as they
    occur in crime scene investigation when using virtual co-location, i.e.
    expert remote crime scene investigators to guide local investigators in
    AR to collaboratively analyse the crime scene.

  \section{Final Product} \label{sec:finalproduct}
    The goal of the game is to solve a puzzle by controlling laser beams
    using mirrors in such a way that a predefined target is hit. The game
    can be played by one or more local players and one or more remote players.

    There are cards present for the local players that represent mirror
    bases. These must be placed on the table, which will be the locations
    for the mirrors. The local players will be able to see the mirrors they
    place through the use of AR technology. Each of the local players will
    only be given a few of the mirror bases needed to solve the puzzle, and
    as such solving the puzzle requires cooperation from all local players.

    The remote players can also see the placed mirrors, and can rotate them
    to influence the path of the laser beam(s). Only by cooperation between
    local players (who can only move the mirror bases) and remote players
    (who can only rotate them) it becomes possible to hit the target and as
    such solve the puzzle.

    The game provides various different types of mirrors with different
    properties, allowing for more complex puzzles. One example of such a
    mirror is a colored mirror, and then require the target is hit with the
    right (combination of) colors. Another way to make puzzles more complex
    is requiring that the players combine beams together to create more
    powerful beams. Other optical components like beam splitters can also be
    introduced.

    The game is designed to stimulate cooperation between the physically
    co-located players and the physically remote player(s). It does so by
    dividing abilities required for solving the puzzles amongst all players
    as follows:

    \begin{itemize}
      \item Physically co-located players each get only a part of the
            mirror bases required to solve the puzzle, requiring input
            from all of these players.
      \item If there are multiple physically remote players, each of
            these players can only rotate a subset of the mirrors, and
            as such input from all physically remote players is
            required for solving the puzzle.
      \item Physically remote players have the ability to rotate
            mirrors while the physically co-located players do not have
            this ability, requiring input from both physically co-located
            as well as physically remote players.
    \end{itemize}

  \section{Software Design Methods} \label{sec:designmethods}
    This section describes the design methods that were used during the
    project. It illustrates the methods that were used to develop and
    coordinate the project during the development phase.

    \subsection{Design Process} \label{ssec:designprocess}
      In designing and implementing the product, it is important that
      requirements can be changed quickly and without much problems. This is
      not because the requirements are likely to change from the client
      side, but because the choice of AR technology may change over the
      course of the project because of technical issues. The available Virtual
      and Augmented Reality glasses are mostly still in development, and as
      such this may affect the technical viability of each device.

      To deal with such changes, we use the Scrum methodology. The Scrum
      methodology describes a set of rules that, amongst others, makes it
      easier to deal with various changes during the development process.
      For a complete description of the rules that the Scrum methodology
      describes, please refer to the Scrum guide, available at
      \url{http://www.scrumguides.org/scrum-guide.html}.

    \subsection{Organization} \label{ssec:organization}
      To be able to simulaneously work on the project without conflicts, we
      use Git as a version control system. The project is stored remotely on
      Github, ensuring the work is efficiently shared between all team members.

      To coordinate and divide the tasks, as well as to maintain the items in
      the Scrum backlog, we use Trello. Trello is an online service that
      provides a dynamic way to organize items in various lists. We created
      lists to keep track of which items were in the backlog, which items were
      being worked on and which items were already done.

      The project is licensed under the terms of the MIT license. We chose
      this license because it allows other developers to learn from this
      project. Additionally, since this project is done as a part of a
      research project, we believe making this project open source may
      help future researchers in the same field. The full terms of the MIT
      license can be found at \url{http://opensource.org/licenses/mit-license.html}

      To ensure the C\# source code in the project meets common coding standards
      (as set by Microsoft), we use the code analysis tools FxCop and StyleCop
      in combination with SonarQube. FxCop performs static code analysis, like
      code complexity and some naming conventions. StyleCop, on the other hand,
      focuses more on code style which includes use of spacing and
      documentation as well as other factors. SonarQube is a platform that
      unifies the reports from these tools and provides a clean overview of the
      combined issues found by FxCop and StyleCop, as well as some simple metrics
      SonarQube has built-in.

    \subsection{Design Architecture} \label{ssec:designarchitecture}
      Because the product is a game and the goal of the project is more
      focused on the game mechanics rather than the underlying engine, we chose
      to use Unity as a starting point. Unity provides a platform-independant
      IDE for developing games, and offers many features commonly used in
      games.

      Using Unity means that the project architecture is bound to the
      loosely coupled component-based architecture that Unity provides,
      although it is possible to include principles from object-oriented
      programming to some extent.
