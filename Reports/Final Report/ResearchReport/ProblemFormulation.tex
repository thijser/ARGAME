\section{Problem Formulation} \label{cha:problem}
	% The problem description is taken from BEPSys. It has been shortened 
	% and slightly modified to fit the context of this document.
	While augmented reality research has grown into a mature field over the 
	last years, the aspects of situational awareness and presence of 
	augmented reality (AR) are still quite open research topics. This 
	project is about designing and implementing a collaborative game to 
	explore the different perception of situational awareness, presence and 
	workload in a physical and an AR environment. The game is to be employed 
	as an approximation of collaboratively solving complex problems, as they 
	occur in crime scene investigation when using virtual co-location, i.e. 
	expert remote crime scene investigators to guide local investigators in 
	AR to collaboratively analyse the crime scene. 
	
	The game needs to support at least three players: At least two players are
	present at the same location (physically co-located), each wearing AR
	glasses. At least one player is physically remote but virtually co-located
	using VR glasses \cite{bepsys}. It should be impossible, or at least
	infeasible, to complete the game without involving the other party. However,
	this constraint could be relaxed to allow a higher playability of the game.
	It would be nice to still be able to play the game if no suitable virtually
	co-located player can be found, for example.
	
	%TODO Add our interpretation to avoid ambiguities
