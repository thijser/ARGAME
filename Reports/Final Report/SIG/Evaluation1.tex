\chapter{SIG Midterm Evaluation} \label{app:sig1}

% Fix hyphenation of Dutch words.
\selectlanguage{dutch}

	De code van het systeem scoort ruim vier sterren op ons 
	onderhoudbaarheidsmodel, wat betekent dat de code bovengemiddeld 
	onderhoudbaar is. De hoogste score is niet behaald door een lagere score
	voor Unit Size, Unit Complexity en Unit Interfacing.

	Voor Unit Size wordt er gekeken naar het percentage code dat bovengemiddeld 
	lang is. Het opsplitsen van dit soort methodes in kleinere stukken zorgt 
	ervoor dat elk onderdeel makkelijker te begrijpen, te testen en daardoor 
	eenvoudiger te onderhouden wordt. Binnen de langere methodes in dit systeem, 
	zoals bijvoorbeeld de 'detector::recognizeMarkers'-methode, zijn aparte 
	stukken functionaliteit te vinden welke ge-refactored kunnen worden naar 
	aparte methodes. Commentaarregels zoals bijvoorbeeld '// Turn into grayscale 
	and threshold to find black and white code' en '// Cut off border' zijn een 
	goede indicatie dat er een autonoom stuk functionaliteit te ontdekken is. Het 
	is aan te raden kritisch te kijken naar de langere methodes binnen dit systeem 
	en deze waar mogelijk op te splitsen.

	Voor Unit Complexity wordt er gekeken naar het percentage code dat bovengemiddeld complex is. Ook hier geldt dat het opsplitsen van dit soort methodes in kleinere stukken ervoor zorgt dat elk onderdeel makkelijker te begrijpen, makkelijker te testen en daardoor eenvoudiger te onderhouden wordt. In dit geval komen de meest complexe methoden ook naar voren als de langste methoden, waardoor het oplossen van het eerste probleem ook dit probleem zal verhelpen.

	Voor Unit Interfacing wordt er gekeken naar het percentage code in units met een bovengemiddeld aantal parameters. Doorgaans duidt een bovengemiddeld aantal parameters op een gebrek aan abstractie. Daarnaast leidt een groot aantal parameters nogal eens tot verwarring in het aanroepen van de methode en in de meeste gevallen ook tot langere en complexere methoden. Wat opvalt in dit systeem is dat zowel in de C\# als in de C++ code soms een Point/Vector abstractie gebruikt wordt, maar dat er ook methoden zijn waar de  parameters 'x' en 'y' los worden doorgegeven. Om het voor toekomstige ontwikkelaars makkelijker te maken om de code te hergebruiken is het aan te raden de abstracties consistent te gebruiken. 

	Daarnaast nog de opmerking dat het goed is om te zien dat de README duidelijk aan geeft dat de 'netlink' code niet zelf geschreven is. Zou het hier nog helpen om duidelijk aan te geven welk versienummer van deze library nu in gebruik is?

	Over het algemeen scoort de code bovengemiddeld, hopelijk lukt het om dit niveau te behouden tijdens de rest van de ontwikkelfase. Als laatste nog de opmerking dat er geen (unit)test-code is gevonden in de code-upload. Het is sterk aan te raden om in ieder geval voor de belangrijkste delen van de functionaliteit automatische tests gedefinieerd te hebben om ervoor te zorgen dat eventuele aanpassingen niet voor ongewenst gedrag zorgen. 