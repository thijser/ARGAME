\documentclass[]{report}
\usepackage[dutch, english]{babel}

\usepackage{url}
\usepackage{cite}
\usepackage{graphicx}
\bibliographystyle{apalike}
\pagenumbering{roman}

\setlength{\parindent}{0pt}
\setlength{\parskip}{5pt plus 2pt minus 1pt}

% Title Page
\title{Augmented Reality mirror game}
\author{Thijs Boumans \and Patrick Kramer \and
        Alexander Overvoorde \and Tim van Rossum}
\begin{document}
\maketitle

\begin{abstract}
This report describes the procedure of development of an augmented reality game
for the TU Delft. This game, called the \emph{Augmented Reality Mirror Game},
is a game that uses augmented reality technology to simulate lasers, mirrors,
a target (or multiple targets), and walls. The goal of this game is to use
mirrors to deflect the laser beam coming from the laser to the target.
%TODO: add description of the chapters.
\end{abstract}
\tableofcontents
\listoffigures

\chapter{Orientation} \label{cha:orientation}

\chapter{Design} \label{cha:design}

\chapter{Implementation} \label{cha:implementation}
	The Bachelor Project course would not be a real Bachelor Project course if 
	there were projects without technical challenges whatsoever regarding the 
	implementation of the final result. Therefore, this chapter elaborates
	on the technical challenges faced during development of the project and
	the solutions that we have developed for these challenges.
	%TODO Highlight technical challenges and elaborate on solutions for these challenges.
	%     This includes (amongst others) how the AR functionality posed a problem and we
	%     implemented a C++ server using OpenCV to take over that work (along with 
	%     synchronizing on a central camera).
	
	% Sections are purely by example. This list might not be entirely correct or complete.
	\section{AR Glasses} \label{sec:arglasses}
		One of the first design choices that we had to make was about what
		AR glass was going to be used with the project. As can be seen in
		our research report, under appendix \ref{app:researchreport},
		there two options to choose from. These were the Oculus and
		the META One. We eventually settled for the META One, because
		of the latency of the Oculus. A pair of META Glasses can be seen in
		figure \ref{fig:metaone}.
		
		\begin{figure}[!ht]
			\centering
			\includegraphics[width=\textwidth]{MetaOneGlasses}
			\caption{A pair of META One glasses.}
			\label{fig:metaone}
		\end{figure}
		
		The challenge with the META One was to get it working in Unity.
		There is a Meta SDK which allows for Unity games to work with
		the META One, but AR itself is still in development (as of
		writing this report), and the META One glasses are experimental
		at best. The SDK that we used first was also very buggy (due to the
		experimental nature of the META One). Also, the META One has a very
		limited field-of-view (the field-of-view was so limited that, during
		one of our first tests with our coach, the coach had to sit back
		to keep everything tracked, which, especially considering that they
		had to move markers as well as track the environment continuously,
		was less than ideal). However, the SLAM tracking built into the
		META allows for game objects to continue to be rendered on a marker
		even if the marker is outside of the view of the META. SLAM tracking
		is explained in subsection \ref{ssec:slamloc}.
		
		\subsection{SLAM tracking} \label{ssec:slamloc}
			SLAM is an acronym, which means Simultaneous Localization And
			Mapping. It stands for a computational problem making a map
			from an unknown environment while updating the location of the
			agent in the same environment. These problems cannot be solved
			independently from each other, as updating a map usually involves
			knowing the location of the agent before any accurate updates
			can be made, and vice versa. Several algorithms have been developed
			for solving this problem, and there is even a platform, called OpenSLAM,
			which contains several open source algorithms which solve this
			problem.
			
			However, the algorithms are beside the point. the real benefit of 
			using SLAM with AR is that SLAM tracking allows the META One to
			render the game objects belonging to a marker while keeping them
			rendered once the marker leaves the field-of-view of the META One.
			Considering that the field-of-view of the META One is not that large
			(As seen in our research report under Appendix \ref{app:researchreport}),
			this is a huge benefit. The META One also has built-in support for
			SLAM tracking of objects, which meant that no time had to be spent
			on developing algorithms.

	\section{Marker Detection} \label{sec:markerdetection}
		...
		
	\section{Synchronization of World State} \label{sec:synchronization}
		Because the game is played by multiple people, the state of the world
        somehow has to be synchronized between all players. To do this, we 
        considered two major options:
        
        The first option was to use the built-in Network View component in 
        Unity. This would allow Unity to take care of most synchronization,
        which in turn could make implementing the synchronization particularly
        easy. However, due to the way the Meta One glasses manipulate the 
        positions and rotations of game objects to fit the orientation of the 
        player's head, synchronizing these positions would result in incorrect
        positions for other players. Instead, a custom serialization method 
        would have to be implemented to undo the manipulation by the Meta One
        and then apply the correct manipulation for each of the other players.
        
        Because of the issues the Network View component would give, a second
        option was considered. This option would introduce the need for a 
        master server that hosted the game, and could provide raw positions
        and rotations of markers exactly as they were placed on the table.
        The only thing left to do would be to move and rotate the objects for 
        each player to match that player's view of the playing area. 
        Additionally, the server approach also solved another recurring issue,
        as highlighted in section \ref{sec:markerdetection}.
        
        In the end, we went for the second option because of the abovementioned
        reasons, but also partly because it allowed us more control over the 
        internal workings of the network functionality and the marker tracking.
        See section \ref{sec:markerdetection} for the details about the marker
        detection performed by the server.
		
	\section{Networking and the OpenCV server} \label{sec:network}
		The game depends on a server application with a master camera. The 
        server application detects the position and rotation of markers in the 
        playing area. It does this through the use of a central so-called master 
        camera, that is position so that the entire playing area is visible from 
        the camera.
        
        The server application is written in C++ and is based on OpenCV and the 
        Qt framework. We decided to implement the server outside of Unity, since 
        Unity does much more than what we need of the server. The server only 
        acts as a way to track all markers even if they aren't seen by any of 
        the players, and to facilitate synchronizing state changes with all 
        players. For example, if a remote player rotates an object in the game, 
        the details about that rotation is sent to the server, which then 
        distributes it to all other players.
        
        A more detailed description about the communication between the Unity 
        clients and the server can be found in paragraph \ref{ssec:communication}.
        The use of the master camera to detect markers is described in paragraph
        \ref{ssec:mastercamera}.
         		
		\subsection{Communication between C\# and C++} \label{ssec:communication}
			Communication between the Unity clients and the OpenCV server 
            happens through the use of sockets. For the Unity side, the Socket 
            facility built into the C\# runtime is used. For the OpenCV server, 
            the TCP Socket facility of the Qt Network module is used for 
            providing a server socket capable of handling multiple clients. 
            
            The protocol used for communication is kept very simple to reduce 
            network load and for simplicity. The protocol consists of a number
            tag indicating the message type followed by the actual message 
            content. To facilitate the features the game provides, the following
            message types are used:
            
            \begin{description}
                \item[Position Update] Sent by the server whenever it detects a 
                                       change in a marker position.
                \item[Position Delete] Sent by the server whenever a marker is
                                       removed from the playing field.
                \item[Rotation Update] Sent by remote players to indicate they 
                                       have rotated an object. This message is 
                                       forwarded to all connected clients by the
                                       server.
                \item[Ping Message]    Sent by the server and clients to indicate 
                                       they are still connected and listening.
            \end{description}
		
		\subsection{The master camera} \label{ssec:mastercamera}
			...
			
	\section{Distinguishable markers} \label{sec:markers}
		...
\chapter{Quality Assurance} \label{cha:qa}
	This chapter explains and describes various quality assurance techniques 
	that were used during the project, to allow us to deliver a product of 
	good quality (regarding both code quality and gameplay quality). It also
	talks a bit about Microsoft Visual Studio first, our editor of choice for
	this project, considering that some QA tools are not available for
	MonoDevelop, the standard editor packaged with Unity.
	
	\section{IDE used for programming} \label{sec:ide}
		The IDE used for programming was Microsoft Visual Studio. Unity has an editor
		of its own available for programming in C\#, which is called MonoDevelop.
		However, the MonoDevelop IDE is really lacking in functionality. It has no
		support for the plugins that we use to check our code. Furthermore, the use
		of MonoDevelop enforces a code style that is incompatible with the style 
		guidelines used by StyleCop (see \ref{sec:codestyle}), 
	
	\section{Testing} \label{sec:testing}
		There are three main types of testing done during the project. These are
		unit testing, integration testing and user testing. Unity has no native
		support for running unit\slash integration tests that have been written,
		but there is a toolkit available for free on the Asset Store, called
		Unity Test Tools,  that does have this support. The extension is
		developed by the Unity team, and can be found on the Unity Asset Store: 
		\url{https://www.assetstore.unity3d.com/en/#!/content/13802}.
		Using this extension, a new menu bar item, called "Unity Test Tools"
		will appear in the main Unity editor. Clicking on this item creates a drop
		down menu with different options, the most important one being the unit test
		runner.
		
		\subsection{C\# Unit Tests} \label{ssec:csharpunittests)}
			Unit tests are written using the NUnit unit testing framework for 
			C\#. NUnit is a test framework which was ported from the Java test 
			framework JUnit, and was created to bring xUnit testing to all .NET 
			languages. Using this framework is also really easy, and a tutorial 
			on how to write unit tests using NUnit can be found using Google. 
			Using Unity Test Tools, all unit tests in the project are listed 
			once one clicks on the subitem "Unit Test Runner". The tests are 
			listed in a new window, and one can run all unit tests by clicking 
			on the "Run All" button at the top. The menu then shows what unit 
			tests have passed or failed, and clicking on a unit test shows what 
			went wrong. A unit testing overview can be seen in \ref{fig:unitytesttools}.
			The UnityTest testing class seen in the overview also displays
			the different statuses of tests in the NUnit framework (passing,
			failing, inconclusive, not executed, and culture specific).
			
			\begin{figure}[!ht]
				\centering
				\includegraphics[width=\textwidth]{UnityTestTools}
				\caption{The Unity Test Tools unit testing screen.}
				\label{fig:unitytesttools}
			\end{figure}
			
		\subsection{C++ Unit Tests} \label{ssec:cplusplusunittests}
			The unit tests for the C++ server code are not written using the
			NUnit framework, as that would not really work. The framework for
			the tests in C++ is the Google Test framework. Google test is, like
			NUnit for C\# and JUnit for Java, an xUnit-based testing framework.
			As such, it also supports assertions, type parameterization, etc.
			Also, it is open source, licensed under the new BSD license.
			
            To implement the testing functionality into the server project, 
            the unit tests are separated into their own subproject, which 
            links to the server and runs the tests. As recommended by the 
            Google Test guide, the Google test framework source and headers 
            are included directly into the unit tests subproject, ensuring 
            that the tests can be compiled and run regardless of the platform 
            or compiler being used.

            Because most of the functionality implemented in C++ involves
            computer vision, we've prepared images so that situations can be
            reliably reproduced for test cases. This allows us to test the
            algorithms under many different lighting conditions at once, for
            instance, which saves us a huge amount of time. Some of the tests
            involve functionality that returns an image as result. We've created
            the image we expect and have written utility functions to check if
            two images are approximately equal.
		
		\subsection{Integration Tests} \label{ssec:integrationtests}
			Integration tests are not done via a formalized test procedure, but 
			rather by creating simple scenes and observing that the subjects of 
			the test work as they should when they are placed in an actual 
			scene. It is also a lot harder to run these tests in a standardized 
			way most of the time.

		\subsection{User Tests}
			Although testing the code is important and it helps ensure that the
			software is working correctly, it doesn't tell us anything about the
			actual usability of the product. Since we have been developing a game, 
			it is especially important that the end users will have fun using the
			product. Properties like these can be evaluated by doing user tests.

			Most of the initial play testing has been done by us, the developers,
			because external parties would not have had a good experience while 
			the game was still in the middle of development. However, as the 
			project was nearing completion, the game could no longer be objectively
			played and tested by the developers. At that point, we were already
			too familiar with the levels and game mechanics to be able to properly
			judge if the concepts are too hard to figure out for new players. By 
			having a lot of other people play the game, we were able to estimate if 
			certain levels were either too easy or too hard, and if all of the 
			different game objects were easy to understand and fun to interact with.

			Finding other computer science students to play the game was not much of 
			a problem, but we also needed to have other people play the game, to get 
			a good overview of how different people perceive the game. This is
			especially important, as computer science students mostly indicated the 
			technical issues and limitations of the Meta One glasses, which were mostly
			already known to us.
			
			%TODO Add results of user testing with non-EEMCS students/people.
		
		\subsection{Code coverage} \label{ssec:codecoverage}
			Code coverage is a metric that can be used to determine how thorough
			the written code has been tested. We have to consider generating 
            code coverage reports for two languages: The C\# Unity project and 
            the C++ server project. 
            
            Generating code coverage reports for the C\# code is unexpectedly 
            difficuly: Even though here are several free software packages 
            available on the Internet that allow for generating code
			coverage reports of NUnit test suites, these are hard to use when 
            combined with unit tests in Unity. The reason behind this is that, 
            to run the unit tests for the game, the Unity Test Tools 
            functionality has to be used (as most tests use instantiation of 
            gameplay objects, something that canonly happen in Unity). This 
            functionality has no way of integrating the NCover or OpenCover 
            software packages. It is possible to integrate these with Microsoft 
            Visual Studio, however it is impossible to run the unit tests from 
            that IDE. As such, we had to manually check if the tests tested all 
            possible branches of the code.
            
            On the other hand, however, analysing code coverage for the C++ 
            project is almost trivial: For GNU compilers, including MinGW, 
            there is a utility called gcov, which measures code coverage of 
            GTest-based unit tests. And when using the Microsoft Visual C++ 
            compiler, it is possible to use Visual Studio to analyse the 
            coverage of the unit tests. The tool used for analyzing C++
            code coverage through tests is called OpenCppCoverage, which
            is compatible with Microsoft Visual Studio 2008 and later.
            It easily measures unit test code coverage, and also does not
            require any extra tools to actually generate the report that
            describes the measuring results. These are immediately afterwards
            given in a HTML file. An example of such a report and its results
            is given in \ref{fig:cpppie}, while an example of which lines of code have
            been analyzed is given in \ref{fig:cpplines}.
            
            \begin{figure}[!ht]
            	\centering
            	\includegraphics[width=\textwidth]{CPPCoveragePie}
            	\caption{The C++ code coverage report.}
            	\label{fig:cpppie}
            \end{figure}
            
            \begin{figure}[!ht]
                \centering
                \includegraphics[scale = 0.5]{CPPCoverageLines}
                \caption{The C++ code coverage report, with covered and uncovered lines.}
                \label{fig:cpplines}
            \end{figure}
			
	\section{Code Style} \label{sec:codestyle}
		We decided to stick to the code style guidelines defined by StyleCop and 
		FxCop, two utilities developed by Microsoft. These utilities check for 
		common programming and code style errors in projects so that these can 
		easily be identified and fixed. 
		
		During the project, we kept the source code style checked by periodically
		dedicating time solely for checking code style and performing code 
		maintenance. This also included writing or improving unit tests and 
		refactoring classes and methods with a relatively high complexity or other 
		issues as indicated by StyleCop and FxCop.
		
	\section{SonarQube} \label{sec:sonarqube}
		For getting a clear overview of the source code quality of the project, as 
		well as the issues indicated by FxCop and StyleCop (see section 
		\ref{sec:codestyle}), we made use of a SonarQube server, hosted on one of 
		our development machines. SonarQube keeps track of issues indicated by 
		the abovementioned tools, and performs various code metrics, like cyclomatic 
		complexity and dependency cycles. Using SonarQube enabled us to spot 
		problematic classes and methods and allowed us to improve the overall 
		structure of the source code of the project. An example of a SonarQube
		overview can be seen in figure \ref{fig:sonarqube}
		
		\begin{figure}[!ht]
			\centering
			\includegraphics[width=\textwidth]{SonarQube}
			\caption{A SonarQube overview.}
			\label{fig:sonarqube}
		\end{figure}
		
		An issue with SonarQube is that, while it has free options for analyzing
		C\# code, it has none of these free options for analyzing C/C++ code,
		because of the preprocessing that can happen in C or C++ (as explained
		in their blog on \url{http://www.sonarqube.org/ccobjective-c-dark-past-bright-future/}).
		For this reason, it is very hard to analyze the C++ code that we use for the
		OpenCV server.
		
	\section{SIG Evaluation} \label{sec:sigevaluation}
		SIG is an acronym which stands for the Software Improvement Group,
		which is a company that is based in Amsterdam. SIG performs code
		analysis to evaluate code quality on a scale from one to five stars,
		and it does so according to the ISO/IEC 25010 model. Code score is 
		based on the maintainability of the code.
		
		During the project, there are two opportunities to deliver the code 
		that we have written to SIG. The first opportunity is at the end of 
		May, and the second is halfway through June. The first opportunity 
		is used as a midterm quality feedback session, for us to get an idea 
		about how good the code is written and what can be done better. The 
		second opportunity is then intended to hand in the improved code, 
		and for us to then get feedback on how well the code has been improved. 
		The improvements in the code also have a weight in the final result 
		for the project.
		
		The midterm and final evaluation reports we received from SIG are added
		as appendices \ref{app:sig1} and \ref{app:sig2}, respectively.
		
	\section{Demo's and playtesting sessions} \label{sec:demos}
		%TODO Write about the demo with the client.
		...

\chapter{Conclusion} \label{cha:conclusion}
    %TODO Focus on the final state of the project.
    %     Include how the project meets the set standards and requirements, 
    %     and elaborate on how it could be improved.
    %
    %     Also reflect on the process as a whole, and indicate bottlenecks 
    %     we encountered during development.
    
    Despite some rocky development, we ended up with a fully playable game in
    week 9 that met all of the must have and almost all of the should have
    requirements. The should have requirements included mirror elevations, but
    we found that this wouldn't work well in practice and replaced this feature
    with portals. Those accomplish the same purpose of transferring a laser beam
    across a wall. We've also done away with color combiners because the results
    of mixing colors would be too complicated for players. The game can support
    this, however, and these color mixers would be easy to implement. The main
    takeaway here is that some of the should haves were not met for gameplay
    reasons rather than technical problems.

    The only component of the game that is still rather lacking is the one that
    gave us the most trouble during development: marker tracking with AR
    glasses. The glasses we are using for the game (Meta One) represent the
    state-of-the-art of augmented reality see-through glasses, but they have a
    field-of-view of just 35 degrees and a marker tracking framerate of about 5.
    The first issue causes players to have a lack of overview unless they take
    quite a few steps back from the table. The second issue causes the mirrors
    to lag behind when the player moves their view, which diminishes the
    immersion of the game. We repeatedly contacted the company behind the
    eyewear, but there were no plans to improve on these issues any time soon.

    Despite these limitations of the augmented reality technology, we have been
    able to produce a fun and complete game to play that can be used to
    experiment with situational awareness like the client desired.

\appendix

\chapter{Research Report} \label{app:researchreport}
\section{Problem Formulation} \label{cha:problem}
	% The problem description is taken from BEPSys. It has been shortened 
	% and slightly modified to fit the context of this document.
	While augmented reality research has grown into a mature field over the 
	last years, the aspects of situational awareness and presence of 
	augmented reality (AR) are still quite open research topics. This 
	project is about designing and implementing a collaborative game to 
	explore the different perception of situational awareness, presence and 
	workload in a physical and an AR environment. The game is to be employed 
	as an approximation of collaboratively solving complex problems, as they 
	occur in crime scene investigation when using virtual co-location, i.e. 
	expert remote crime scene investigators to guide local investigators in 
	AR to collaboratively analyse the crime scene. 
	
	The game needs to support at least three players: At least two players are
	present at the same location (physically co-located), each wearing AR
	glasses. At least one player is physically remote but virtually co-located
	using VR glasses \cite{bepsys}. It should be impossible, or at least
	infeasible, to complete the game without involving the other party. However,
	this constraint could be relaxed to allow a higher playability of the game.
	It would be nice to still be able to play the game if no suitable virtually
	co-located player can be found, for example.
	
	%TODO Add our interpretation to avoid ambiguities

\chapter{Problem Analysis} \label{cha:analysis}
	% Includes the challenges surrounding AR and Networking.
	% Also motivates the choices made.
	This chapter provides an analysis of the problem description of problems
	and challenges that may arise during development. It provides an analysis
	of the problems and possible solutions that can be used in order to
	solve these problems.

	One of the core challenges of the project is the use of Augmented Reality
	(AR) technology. An analysis of the available options to implement this
	functionality is given in section \ref{sec:ar}. Another important challenge
	is improving situational awareness, which is discussed in section
	\ref{sec:awareness}. The last challenge is the creation of interdependence
	between players in such a way that requires collaboration from all players.
	This challenge is analyzed in section \ref{sec:interdependence}.

	Lastly, a conclusion based on the analyses of these challenges is given in
	section \ref{sec:analysisconclusion}.

	\section{Augmented Reality (AR) Functionality} \label{sec:ar}
		% AR is a core element in the project. As such, we need to compare
		% various AR hardware devices and corresponding ways to implement
		% AR functionality for each device.
		Augmented Reality (AR) is a core aspect of the problem formulation.
		As such, careful analysis has to be done as to how the AR functionality
		can be best implemented to fully address the context of this project.

		We consider two choices for implementing AR functionality: The META One,
		an optical see-through device (\ref{ssec:metaone}), and the Oculus Rift
		Virtual Reality glasses in conjunction with mounted cameras
		(\ref{ssec:oculusrift}).

		% Hardware devices to consider:
		\subsection{META One} \label{ssec:metaone}
			%   - META One (as indicated by the BEPSys project page)
			%       - Has a limited Field-of-View (around 35 degrees)
			%          - May interfere with the experience of the game.
			%       - Optical see-through glasses means AR works out of the box.

			The META One glasses are optical see-through glasses. Optical
			see-through glasses work by projecting a virtual image on top of the
			world you see, effectively implementing a 3D AR exprience.

			Because the META One is an optical see-through device that also
			features motion tracking, AR can be implemented simply by
			projecting an image against a black background to the glasses.

			A big drawback of the available META One glasses is their
			Field-of-View, which is 35 degrees. This Field-of-View is way lower
			than the Field-of-View of a person, which may have a negative impact
			on the game experience.

		\subsection{Oculus Rift} \label{ssec:oculusrift}
			%   - Oculus Rift + mounted cameras (http://oculusvr.com/)
			%       - High Field-of-View (100 degrees)
			%       - VR glasses, so need to project the real world using cameras.
			%          - Limited resolution creates blurriness.
			%          - Projection can be done from within Unity
			%          - Potentially requires a lot of calibration
			% 		- Using Oculus Rift means we need to implement AR functionality
			%         ourselves (as optical see-through often has this built-in).
			%         AR Libraries to consider:
			We've built a camera rig for the Oculus Rift that can be used to
			turn it into an augmented reality device. To detect the markers and
			render objects on them in Unity, there are several libraries
			available. Each of these will be discussed in the next sections.

			Oculus offers an SDK for Unity that makes it easy to integrate a
			game with the Rift. The challenge that we'll be facing during
			development is to properly integrate this SDK with the augmented
			reality libraries. Each of the frameworks try to take control of the
			camera in different ways and it's easy to get conflicts there.
			Getting the Rift see-through functionality working in Unity on its
			own and the augmented reality functionality on its own is not a
			challenge.

			\subsubsection{Vuforia} \label{sssec:vuforia}
				%   - Vuforia (http://vuforia.com/ and http://developer.vuforia.com/)
				%       - Includes integration with Unity
				Vuforia is a framework by Qualcomm that allows you to create
				arbitrary markers, import them into Unity and place objects onto
				them. You can then select a webcam and have it render the camera
				images with 3D objects projected onto the markers. It's very
				easy to use and has built-in support for virtual reality
				solutions like GearVR. The tracking quality is very good and
				stable, even with low quality markers (with few color transitions).

				Unfortunately it currently only works with the 32-bit version of
				Unity. It also lacks support for the Oculus Rift on the desktop,
				which means that we'll have to build that functionality ourselves.

			\subsubsection{Unity AR Toolkit (UART)} \label{sssec:uart}
				%   - Unity AR Toolkit (UART) (https://research.cc.gatech.edu/uart/content/contents/)
				%       - Source code and demos hosted on SourceForge (http://sourceforge.net/projects/uart/)
				%       - Seems to be a research project
				%       - Seems easy to use (comes with examples)
				%       - Built for Unity
				%       - Last change to SVN repo was in 2011.


			\subsubsection{Metaio} \label{sssec:metaio}
				%   - Metaio (http://www.metaio.com/)
				%       - Mainly oriented towards mobile phones, so may not be suitable for this project

		%   - ... <Add more as needed>

	\section{Situational Awareness} \label{sec:awareness}
		% Improving situational awareness is part of the main goal of the project.
		% We should indicate the steps needed to achieve this, which can be based
		% on (possibly) a large range of scientific articles.
		This project is about exploring the different perception of situational
		awareness, presence and workload in a physical and an AR environment
		(see chapter \ref{cha:problem}). As such, situational awareness plays a
		key role in this project.

		Before considering how situational awareness plays a role in this project,
		it is important to define exactly what situational awareness means.
		According to \cite{endsley}, situational awareness is defined as "the
		perception of the elements in the environment within a volume of time and
		space, comprehension of their meaning, and the projection of their status in
		the near future". In other words, situational awareness means to
		fully understand the situation, and be able to predict what is going to
		happen next. This also includes understanding any risks the situation brings.


	\section{Interdependence between players} \label{sec:interdependence}
		% Creating interdependence between players requires them to work together.
		% This can be done in several ways. We need to elaborate on the various ways
		% in which this can be achieved.
		The problem formulation states that the game is to be employed as an
		approximation of collaboratively solving complex problems. In order to
		motivate players of the game to collaborate, there is a need to create
		a form of interdependence amongst the players. One way to do this is to
		create an asymmetry between either the information that the players have or
		an asymmetry of abilities, as explained in the following subsection. 
		\subsection{Asymmetry of abilities}
			The main reason to co-operate is the asymmetry of abilities between
			the players involved. For example: physically co-located players can
			alter the game world, while virtually co-located players can guide
			characters to a certain goal utilizing the altered game environment.
			One thing to note is that a "puppet master" scenario should be avoided.
			This scenario happens when one player can do everything except for a
			few required tasks, and uses the other players to execute these tasks.
			In this case, the other players will have less involvement with the
			shared goal, and the amount of co-operation will go down.
		\subsection{Asymmetry of information}
			Asymmetry of information could be used as another reason for the players
			to co-operate. It means that both types of players (both the physically
			co-located and the virtually co-located) have different, separate, parts
			of the information required to complete the game. in this case, a 
			"puppet master" scenario should also be avoided. Such a scenario can occur
			here when one type of player has nearly all of the information or can
			infer nearly all information.
		
	\section{Virtual Co-location} \label{sec:virtualcolocation}
		Establishing virtual co-location is required to allow physically remote players
		to play the game together. As such, both the virtualization of the game world and
		the networking are considered in virtually co-locating physically remote players.
		Unity has multiplayer support, because of its master server to handle multiplayer
		games, but the server could be down at times. There are tutorials on the internet
		to create a basic multiplayer game that uses the master server to handle requests.
		These tutorials can be used to implement our own multiplayer support.
		% Allowing phsyically remote players to play the game, we need to establish
		% some idea of virtual co-location. This includes the virtualization of the
		% game world as well as the networking functionality required to establish
		% the actual connection.

	\section{Analysis Conclusions} \label{sec:analysisconclusion}
		\section{Analysis Conclusions} \label{sec:analysisconclusion}
  % Provides the coices we made for the abovementioned problems along with a short
  % motivation based on the above analysis.

		% Provides the coices we made for the abovementioned problems along with a short
		% motivation based on the above analysis.

\chapter{Proposed Solutions} \label{cha:solution}
	% A description of the proposed solution, including gameplay elements.
	% The description is taken from the Product Plan. This section also
	% motivates why the proposed solution is a good solution.

\section{Laser mirror game} \label{sec:lasergame}
	The goal of the game is to solve a puzzle by controlling laser beams
	using mirrors in such a way that a predefined target is hit. The game
	can be played by one or more local players and one or more remote players.

	There are cards present for the local players that represent mirror
	bases. These must be placed on the table, which will be the locations
	for the mirrors. The local players will be able to see the mirrors they
	place through the use of AR technology. Each of the local players will
	only be given a few of the mirror bases needed to solve the puzzle, and
	as such solving the puzzle requires cooperation from all local players.

	The remote players can also see the placed mirrors, and can rotate them
	to influence the path of the laser beam(s). Only by cooperation between
	local players (who can only move the mirror bases) and remote players
	(who can only rotate them) it becomes possible to hit the target and as
	such solve the puzzle.

	The game provides various different types of mirrors with different
	properties, allowing for more complex puzzles. One example of such a
	mirror is a colored mirror, and then require the target is hit with the
	right (combination of) colors. Another way to make puzzles more complex
	is requiring that the players combine beams together to create more
	powerful beams. Other optical components like beam splitters can also be introduced.

	The game is designed to stimulate cooperation between the physically
	co-located players and the physically remote player(s). It does so by
	dividing abilities required for solving the puzzles amongst all players
	as follows:

	\begin{itemize}
		\item Physically co-located players each get only a part of the
		      mirror bases required to solve the puzzle, requiring input
		      from all of these players.
		\item If there are multiple virtually co-located players, each of
		      these players can only rotate a subset of the mirrors, and
		      as such input from all virtually co-located players is
		      required for solving the puzzle.
		\item Virtually co-located players have the ability to rotate
		      mirrors while the physically co-located players do not have
		      this ability, requiring input from both physically as well
		      as virtually co-located players.
	\end{itemize}

	Because of this division of abilities, there is an interdependence
	between all players (see section \ref{sec:interdependence}), regardless of
	whether these players are physically co-located or not. This replicates the
	interdependence that exists in the crime scene example as given in the
	problem formulation (see chapter \ref{cha:problem}).

\section{Platformer} \label{sec:platformer}
	The game starts with the avatar of the remote player appearing
	on the table and a goal appearing above the table. The local
  players have a pile of blocks, each of these blocks exists both
  in the virtual and real world.

	Inside the virtual world a number of pre-existing structures and
	obstacles exist making it harder for the remote player to move around.
	The local and remote player must work together in order to get the
	virtual avatar to the goal.

	Team work is heavily encouraged due to the interdependence between the
	players: There is information asymmetry (section \ref{ssec:information})
	because the remote player can see some pre-existing blocks the local
	player cannot see, and the fact that local players can place new blocks
	in the game world provides ability asymmetry (section \ref{ssec:ability}).

\section{FPS Survival game} \label{sec:fpssurvival}
	The goal of the game is to protect a virtual structure from enemies. These
	enemies will come from multiple sides to attack the structure while the
	remote players have to stop them. For doing so, they require the aid of the
	local players, who can place blocks (similar to the platformer game, section
	\ref{sec:platformer}) to block the path of the enemies.

	The local players can block the path of the enemies, but they cannot defeat
	them. The remote players, who view the game from a first-person perspective,
	have to defeat the enemies. Since the perspective of the remote players is
	limited (also partly because of the blocks placed by local players), they
	require information from the local players, who view the scene from above
	and as such have a good view of the entire situation. This creates
	information asymmetry between the players (section \ref{ssec:information}).

	To keep the game challenging, the enemies will grow stronger over time,
	requiring the cooperation between the local and remote players to improve
	as well. Because of the fact that local players can only defeat the enemies,
	and the remote players can block their path, there is a form of ability
	asymmetry between the local and remote players (section \ref{ssec:ability}).

\section{Tower defense} \label{sec:towerdefense}
	The local players start off with a number of towers each
	which have certain strengths and abilities. They must place
	these towers on a table in front of them. What they cannot,
	but the remote player can see is what paths a number of
	hostile entities will take in order to attack their base.
	They must therefore cooperate to place the towers on the right
	positions to be able to defeat the enemies before they escape
	by communicating the ideal locations of the
	towers and paths that the hostile entities will take.

	This game idea is relatively simple, and because of the simplicity,
	complexity can easily be added to make the game more engaging. For
	example, towers could be upgradeable, utilizing resources that
	could be gained either by defeating individual enemies or by defeating
	a wave of enemies. The idea behind these resources is that they are
	shared between players. They could then be used to upgrade
	tower types. As these resources are shared, and they can only be
	used once, players must work together to choose the best upgrade
	available in the given scenario. Enemies should also get procedurally
	stronger because of the increased capabilities of the towers. Resources
	could also be put towards research for new tower types, which players
	could then use.

	Another way to add extra necessary complexity would be through
	introducing an experience system, and to grant towers some experience based
	on what enemy they have defeated. These towers would also over time get
	stronger. Bloons Tower Defense, a game found on the internet, is a tower
	defense system that uses both resources and experience, to show that these
	could also be combined. New towers would then be unlocked over time, after
	completing a certain amount of waves, instead of using resources.

	The problem with this idea is that it will very likely create a
	\emph{puppet master} scenario where the remote player takes charge
	of the whole situation and the other players do net get a
	say in what is going to happen. This could be avoided by a
	number of extra abilities. This game shows heavy
	information and ability asymmetry.

\section{Minesweepers}
	This concept is based on the classic game of minesweeper. There's a grid
	where some of the squares have mines under them. Players each start at a
	random position and have to place flags on locations of mines while avoiding
	mines. The remote player is the only one who can see the numbers around
	squares, so he has to give instructions to the players on the field.

	The difference compared to the classic version of the game is that it's all
	physical. Local players walk around on the field, where they have to take
	careful steps to avoid triggering a mine. This turns the game into some sort
	of Twister variant where people can use special moves to quickly traverse
	the field and find all the mines. Local players co-operate by dividing the
	field into sections that they'll clear in parallel.

	The problem with this concept is the inherent \emph{puppet master} phenomenon.
	Communication between the remote player and local players is very one-sided.
	Local players just receive commands where to step and where to plant flags
	and don't really have any input into the game themselves. Even if that
	problem was solved, the lack of cooperation aspects between local players
	would also be a big problem. Local players don't really have any reason to
	interact with each other. One way to solve that problem would be to require
	multiple players to place a flag.

	There are also some technical challenges with this concept. It requires that
	local players always know their exact position on a large field that they're
	inside of, which would require a lot of markers. Next, the position of their
	legs would have to be determined somehow to ensure that they're not stepping
	on a mine. Finally, a space large enough for a field would likely have to be
	found outside and sunlight doesn't play well with augmented reality devices.
	The game could be tranformed into an indoor variant instead, using a board
	and pawns, but this does not solve the one-sided communication problem.

\section{The chosen idea} \label{sec:chosenidea}
	The idea that is chosen is the mirror game idea. The reason for this
	decision is that it is a relatively simple idea, it can create hard
	to solve puzzles, and extra complexity can easily be added. It also
	prevents a \emph{puppet master} scenario by giving both types of players
	about half of the abilities required for solving the game.

	The game offers a strong interdependence between players. This, together
	with the ability asymmetry caused by the separation between moving mirrors
	and rotating them, causes a high need for collaboration between the local
	and remote players. As such, this concept comes closest to the problem
	formulation in creating a collaborative game between physically remote
	players.

\chapter{Conclusion} \label{cha:conclusion}
    %TODO Focus on the final state of the project.
    %     Include how the project meets the set standards and requirements, 
    %     and elaborate on how it could be improved.
    %
    %     Also reflect on the process as a whole, and indicate bottlenecks 
    %     we encountered during development.
    
    Despite some rocky development, we ended up with a fully playable game in
    week 9 that met all of the must have and almost all of the should have
    requirements. The should have requirements included mirror elevations, but
    we found that this wouldn't work well in practice and replaced this feature
    with portals. Those accomplish the same purpose of transferring a laser beam
    across a wall. We've also done away with color combiners because the results
    of mixing colors would be too complicated for players. The game can support
    this, however, and these color mixers would be easy to implement. The main
    takeaway here is that some of the should haves were not met for gameplay
    reasons rather than technical problems.

    The only component of the game that is still rather lacking is the one that
    gave us the most trouble during development: marker tracking with AR
    glasses. The glasses we are using for the game (Meta One) represent the
    state-of-the-art of augmented reality see-through glasses, but they have a
    field-of-view of just 35 degrees and a marker tracking framerate of about 5.
    The first issue causes players to have a lack of overview unless they take
    quite a few steps back from the table. The second issue causes the mirrors
    to lag behind when the player moves their view, which diminishes the
    immersion of the game. We repeatedly contacted the company behind the
    eyewear, but there were no plans to improve on these issues any time soon.

    Despite these limitations of the augmented reality technology, we have been
    able to produce a fun and complete game to play that can be used to
    experiment with situational awareness like the client desired.

\chapter{Product plan} \label{app:productplan}
\section{Game concept}

The goal of the game is to solve a puzzle by controlling laser beams using mirrors
in such a way that a predefined target is hit. The game can be played by 
one or more local players and one or more remote players.

There are cards present for the local players that represent mirror bases. These 
must be placed on the table, which will be the locations for the mirrors. The local
players will be able to see the mirrors they place through the use of AR technology.
Each of the local players will only be given a few of the mirror bases needed to solve
the puzzle, and as such solving the puzzle requires cooperation from all local players.

The remote players can also see the placed mirrors, and can rotate them to influence 
the path of the laser beam(s). The remote players is also the only one who can see the
obstacles that are predefined for that particular puzzle. Only by cooperation between 
local players (who can only move the mirror bases) and remote players (who can only 
rotate them) it becomes possible to hit the target and as such solve the puzzle.

The game provides various different types of mirrors with different properties, 
allowing for more complex puzzles. One example of such a mirror is a colored mirror, 
and then require the target is hit with the right (combination of) colors. Another 
way to make puzzles more complex is requiring that the players combine beams together
to create more powerful beams. Other optical components like beam splitters can
also be introduced.

\subsection{Requirements}

\paragraph{Must haves}
\begin{itemize}
	\item A light source and a target must be visible when the game starts.
	\item A light source must emit a light beam in a predefined direction.
	\item A light beam must reflect against a mirror when it hits one.
	\item A local player must be able to see mirrors over the cards using AR technology.
	\item A local player must be able to move the mirrors by moving the cards.
	\item A remote player must be able to see the mirrors in the same positions and 
		  orientations as the local players.
	\item A remote player must be able to rotate the mirrors.
\end{itemize}

\paragraph{Should haves}
\begin{itemize}
	\item TODO
\end{itemize}

\paragraph{Could haves}
\begin{itemize}
	\item TODO
\end{itemize}

\paragraph{Won't haves}
\begin{itemize}
	\item TODO
\end{itemize}
\section{Approach}

This section covers development details that aren't directly related to the
gameplay itself, such as software engineering practices and guidelines about
client meetings.

\subsection{Technical details}

Although AR glasses have been provided to us, their field-of-view is very
limited and is not suitable for most of our concepts. We experienced ourselves
that the limited fov causes a lot of problems. Instead, we're going to try to
use an Oculus Rift and cameras to create our own high fov augmented reality
glasses.

The Unity game engine is going to be used to render the in-game objects like
mirrors and laser beams over the camera image.

{\color{red} TODO: Find suitable AR library and write about it here}

\subsection{Software engineering}

Using the Unity game engine means that we'll use a loosely coupled
component-based architecture from the start. Unity has built-in unit and
integration testing systems, which we'll make extensive use of in development.
We'll also maintain UML diagrams of the architecture to keep an overview of how
everything works, and to plan integration of new features.

We'll make use of the Scrum software development methodology to plan new
features and to help stick to the schedule. As is expected, there will be a
playable demo at the end of each weekly sprint. As soon as the game has all the
basic ingredients to be fun to play, we'll find users to play test on a regular
basis and collect feedback from them to improve the gameplay.

The version control software used is Git. The project is on GitHub. Git uses a master branch (the 
basic branch which contains the project), as well as multiple other, user-defined branches.
The branch structure of Git allows a team to work in separate branches, to merge the branches 
into the master branch later when that part of the project is done. This is done by making 
a pull request, defining what was changed, and afterwards the branch can be merged with the master branch.
The way branches work also allow other team members to do code reviews of a branch, as all changed/added
code can be seen and commented on. This makes Git superior to SVN.

\subsection{Guidelines}

Here is a list of rules that help prevent problems during development.

{\color{red} NOTE: These are just ideas right now}

\begin{itemize}
	\item Meet with the client and coach every week to show a working demo
	\item Add tests as soon as new methods are added to verify that they work
	\item Have integration tests for common scenarios to reduce the need for user testing
	\item For C\# code, we adopt the guidelines posed by Microsoft
		  (See \url{https://msdn.microsoft.com/en-us/library/ms229042.aspx})
\end{itemize}
\section{Planning}

The first two weeks represent the research phase. In this phase we will find a suitable augmented reality (AR) library for Unity and prepare an Oculus Rift for AR use with cameras. We'll also design an architecture for the game that covers the marker detection, networking and gameplay mechanics. All of this information will be described in the research report handed in on May 1st.

Main development commences after this phase and is organized in two week sprints. The table below describes the goals per sprint, which will serve as a helpful reference to stay on schedule during development.

\begin{tabular}{|l|c|r|}
	\hline
	Weeks & Deadline & Goal \\ \hline
	4.1 + 4.2 & May 1st & Research report described above \\ \hline
	4.3 + 4.4 & May 15th & ??? \\ \hline
	4.5 + 4.6 & May 29th & ??? \\ \hline
	4.7 + 4.8 & June 12th & ??? \\ \hline
	4.9 + 4.10 & June 26th & ??? \\ \hline
\end{tabular}

% SIG evaluations
\chapter{SIG Midterm Evaluation} \label{app:sig1}

% Fix hyphenation of Dutch words.
\selectlanguage{dutch}

	De code van het systeem scoort ruim vier sterren op ons 
	onderhoudbaarheidsmodel, wat betekent dat de code bovengemiddeld 
	onderhoudbaar is. De hoogste score is niet behaald door een lagere score
	voor Unit Size, Unit Complexity en Unit Interfacing.

	Voor Unit Size wordt er gekeken naar het percentage code dat bovengemiddeld 
	lang is. Het opsplitsen van dit soort methodes in kleinere stukken zorgt 
	ervoor dat elk onderdeel makkelijker te begrijpen, te testen en daardoor 
	eenvoudiger te onderhouden wordt. Binnen de langere methodes in dit systeem, 
	zoals bijvoorbeeld de 'detector::recognizeMarkers'-methode, zijn aparte 
	stukken functionaliteit te vinden welke ge-refactored kunnen worden naar 
	aparte methodes. Commentaarregels zoals bijvoorbeeld '// Turn into grayscale 
	and threshold to find black and white code' en '// Cut off border' zijn een 
	goede indicatie dat er een autonoom stuk functionaliteit te ontdekken is. Het 
	is aan te raden kritisch te kijken naar de langere methodes binnen dit systeem 
	en deze waar mogelijk op te splitsen.

	Voor Unit Complexity wordt er gekeken naar het percentage code dat bovengemiddeld complex is. Ook hier geldt dat het opsplitsen van dit soort methodes in kleinere stukken ervoor zorgt dat elk onderdeel makkelijker te begrijpen, makkelijker te testen en daardoor eenvoudiger te onderhouden wordt. In dit geval komen de meest complexe methoden ook naar voren als de langste methoden, waardoor het oplossen van het eerste probleem ook dit probleem zal verhelpen.

	Voor Unit Interfacing wordt er gekeken naar het percentage code in units met een bovengemiddeld aantal parameters. Doorgaans duidt een bovengemiddeld aantal parameters op een gebrek aan abstractie. Daarnaast leidt een groot aantal parameters nogal eens tot verwarring in het aanroepen van de methode en in de meeste gevallen ook tot langere en complexere methoden. Wat opvalt in dit systeem is dat zowel in de C\# als in de C++ code soms een Point/Vector abstractie gebruikt wordt, maar dat er ook methoden zijn waar de  parameters 'x' en 'y' los worden doorgegeven. Om het voor toekomstige ontwikkelaars makkelijker te maken om de code te hergebruiken is het aan te raden de abstracties consistent te gebruiken. 

	Daarnaast nog de opmerking dat het goed is om te zien dat de README duidelijk aan geeft dat de 'netlink' code niet zelf geschreven is. Zou het hier nog helpen om duidelijk aan te geven welk versienummer van deze library nu in gebruik is?

	Over het algemeen scoort de code bovengemiddeld, hopelijk lukt het om dit niveau te behouden tijdens de rest van de ontwikkelfase. Als laatste nog de opmerking dat er geen (unit)test-code is gevonden in de code-upload. Het is sterk aan te raden om in ieder geval voor de belangrijkste delen van de functionaliteit automatische tests gedefinieerd te hebben om ervoor te zorgen dat eventuele aanpassingen niet voor ongewenst gedrag zorgen. 
\chapter{SIG Final Evaluation} \label{app:sig2}

% Fix hyphenation of Dutch words.
\selectlanguage{dutch}

	In de tweede upload zien we dat het codevolume met ongeveer 50 procent is gestegen, terwijl de score voor onderhoudbaarheid ongeveer gelijk is gebleven.
	
	Bij Unit Size en Unit Complexity valt op dat de deelscores voor de C++ code zijn gestegen, maar voor de C\# code zijn gedaald. Die daling wordt vernamelijk veroorzaakt door een aantal lange en complexe methodes die jullie sinds de vorige upload hebben toegevoegd. Voorbeelden zijn MarkerTransformer.Update en ClientSocket.ReadMessage. Aan de namen van die methodes kun je al zien dat ze erg veel doen, en het opsplitsen in deelproblemen zou deze methodes beter onderhoudbaar en vooral beter testbaar maken.
	
	Bij Unit Interfacing zien we geen structurele verbetering voor zowel de C\# als de C++ code. Zoals bij de eerste upload al aangegeven wijst dit vaak op een gebrek aan abstractie. In MarkerTracker komen bijvoorbeeld vier methodes voor die allemaal dezelfde lijst van vier parameters aan elkaar doorgeven.
	
	Tot slot is het goed om te zien dat jullie sinds de vorige upload unit testcode hebben toegevoegd. De hoeveelheid testcode is nog wel vrij beperkt, maar dit is wel te verklaren aangezien jullie vrij laat begonnen zijn.
	
	Uit deze observaties kunnen we concluderen dat de aanbevelingen van de vorige evaluatie deels zijn meegenomen in het ontwikkeltraject.
	
	\selectlanguage{english}

\pagebreak
\addcontentsline{toc}{chapter}{Bibliography}
\bibliography{Bibliography}

\end{document}
