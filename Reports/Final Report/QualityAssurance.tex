\chapter{Quality Assurance} \label{cha:qa}
	This chapter explains and describes various quality assurance techniques 
	that were used during the project, to allow us to deliver a product of 
	good quality (regarding both code quality and gameplay quality).
	%TODO Explain how we did Quality Assurance during the project.
	%     This includes a description of FxCop, StyleCop and SonarQube,
	%     but also the use of UnityTestTools and other metrics.
	%     Also include a reference to the SIG evaluation (the actual
	%     evaluation could be an appendix).

	\section{Testing} \label{sec:testing}
	There are two main types of testing done during the project. These are
	unit testing and integration testing. Unity has no native support for
	running unit/integration tests that have been written, but there is a toolkit
	available for free on the Asset Store, called Unity Test Tools, that does
	have this support. The extension is developed by the Unity team, and can be 
	found here: \url{https://www.assetstore.unity3d.com/en/#!/content/13802}.
	Using this extension, a new menu bar item, called "Unity Test Tools"
	will appear in the main Unity editor. Clicking on this item creates a drop
	down menu with different options, the most important one being the unit test
	runner.
	
	Unit tests are written using the NUnit unit testing framework for C\#. NUnit
	is a test framework which was ported from the Java test framework JUnit, and
	was created to bring xUnit testing to all .NET languages. Using this framework
	is also really easy, and a tutorial on how to write unit tests using NUnit
	can be found using Google. Using Unity Test Tools, all unit tests in the
	project are listed once one clicks on the subitem "Unit Test Runner".
	The tests are listed in a new window, and one can run all unit tests by
	clicking on the "Run All" button at the top. The menu then shows what unit tests
	have passed or failed, and clicking on a unit test shows what went wrong.
	
	Integration tests are not done via a formalized test procedure, but rather by
	creating simple scenes and observing that the subjects of the test work
	as they should when they are placed in an actual scene. It is also a lot
	harder to run these tests in a standardized way most of the time. 
		
	\section{Code Style} \label{sec:codestyle}
		...
		
	\section{SonarQube} \label{sec:sonarqube}
		...
		
	\section{SIG Evaluation} \label{sec:sigevaluation}
		...
		
	\section{Demo's and playtesting sessions} \label{sec:demos}
