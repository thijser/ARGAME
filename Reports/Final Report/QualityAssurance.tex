\chapter{Quality Assurance} \label{cha:qa}
	This chapter explains and describes various quality assurance techniques 
	that were used during the project, to allow us to deliver a product of 
	good quality (regarding both code quality and gameplay quality). It also
	talks a bit about Microsoft Visual Studio first, our IDE of choice for
	this project, considering that some QA tools are not available for
	MonoDevelop, the standard IDE packaged with Unity.
	%TODO Explain how we did Quality Assurance during the project.
	%     This includes a description of FxCop, StyleCop and SonarQube,
	%     but also the use of UnityTestTools and other metrics.
	%     Also include a reference to the SIG evaluation (the actual
	%     evaluation could be an appendix).
	\section{IDE used for programming} \label{sec:ide}
	The IDE used for programming was Microsoft Visual Studio. Unity has an IDE
	of its own available for programming in C\#, which is called MonoDevelop.
	However, the MonoDevelop IDE is really lacking in functionality. It has no
	support for the plugins that we use to check our code. 
	
	\section{Testing} \label{sec:testing}
	There are two main types of testing done during the project. These are
	unit testing and integration testing. Unity has no native support for
	running unit/integration tests that have been written, but there is a toolkit
	available for free on the Asset Store, called Unity Test Tools, that does
	have this support. The extension is developed by the Unity team, and can be 
	found here: \url{https://www.assetstore.unity3d.com/en/#!/content/13802}.
	Using this extension, a new menu bar item, called "Unity Test Tools"
	will appear in the main Unity editor. Clicking on this item creates a drop
	down menu with different options, the most important one being the unit test
	runner.
	
	Unit tests are written using the NUnit unit testing framework for C\#. NUnit
	is a test framework which was ported from the Java test framework JUnit, and
	was created to bring xUnit testing to all .NET languages. Using this framework
	is also really easy, and a tutorial on how to write unit tests using NUnit
	can be found using Google. Using Unity Test Tools, all unit tests in the
	project are listed once one clicks on the subitem "Unit Test Runner".
	The tests are listed in a new window, and one can run all unit tests by
	clicking on the "Run All" button at the top. The menu then shows what unit tests
	have passed or failed, and clicking on a unit test shows what went wrong.
	
	Integration tests are not done via a formalized test procedure, but rather by
	creating simple scenes and observing that the subjects of the test work
	as they should when they are placed in an actual scene. It is also a lot
	harder to run these tests in a standardized way most of the time. 
	
		\subsection{Code coverage} \label{ssec:codecoverage}
		Code coverage is a metric that can be used to determine how thorough
		the written code has been tested. There are several free software
		packages available on the Internet that allow for generating code
		coverage reports of NUnit test suites, such as NCover or PartCover.
		These, however, are hard to use when combined with unit tests in
		Unity. The reason behind this is that, to run the unit tests for
		the game, the Unity Test Tools functionality has to be used (as most
		tests use instantiation of gameplay objects, something that can
		only happen in Unity). This functionality has no way of integrating
		the NCover or PartCover software packages. It is possible to integrate
		these with Microsoft Visual Studio, however it is impossible to run
		the unit tests from that IDE. As such, we had to manually check
		if the tests tested all possible branches of the code.
		
	\section{Code Style} \label{sec:codestyle}
		...
		
	\section{SonarQube} \label{sec:sonarqube}
		...
		
	\section{SIG Evaluation} \label{sec:sigevaluation}
		...
		
	\section{Demo's and playtesting sessions} \label{sec:demos}
