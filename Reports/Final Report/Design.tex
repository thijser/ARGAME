\chapter{Design} \label{cha:design}
	This chapter explains the design of the system. This includes: back-end
	design, model/graphical design, and main activities.

	\section{Main activities} \label{sec:mainactivities}
		There are several main activities in the system, corresponding to the two
		main user types of the system. These user types are both the physically co-
		located	players as well as the physically remote players. The activities
		are as follows:

		\subsection{User wants to start a game} \label{ssec:userstartgame}
			A user (or player) wants to start a new game. This user can be either
			a physically co-located player or a physically remote player. In
			order to start a new game, the user has to click the correct buttons
			on the start screen. Clicking the buttons can only happen on a laptop,
			but that should not be a problem considering the user needs to be near
			their laptop anyway to view the game world from an Oculus Rift or a
			META One, the two tools required to play the game with. When the game
			is started, other users, both physically co-located and physically
			remote, can join the same session to play the game together.

		\subsection{Physically remote user wants to join a started game}
				\label{ssec:remotejoingame}
			A physically remote player wants to join an active game. Before this can
			happen, a game has to be started first. As mentioned before, both physically
			co-located players and physically remote players can start a game session
			which other players can join.

	\section{Back-end design} \label{sec:backenddesign}
		The system is composed of three main parts: The Laser mechanics, the network
		functionality and the projection to 3D glasses. For this purpose, we divided
		the code over three namespaces, named "Laser", "Network" and
		"Projection", respectively.

		The "Laser" namespace is responsible for drawing the Laser beams and
		providing the interactions of laser beams with the other game objects.

		The "Network" namespace is responsible for synchronizing the game state
		between all connected players.

		The "Projection" namespace is responsible for providing the projection to
		the VR glasses. This namespace provides the functionality required to project
		the game world to the VR glasses.

	\section{Game elements} \label{sec:graphicaldesign}
		For designing the 3D models, we used Blender. Blender is a free application
		for 3D modeling, and Unity natively support Blender models (provided Blender
		is installed on the system). We have chosen for a light looking style featuring 
		nature inspired modeling and gold and crystal based materials. The light modeling 
		style causes slight miss alignments with the ground to be less noticeable and
		makes the lack of feedback from moving a card feel less odd. The crystals and gold 
		just feels good in combination with the beams of light.
		The following sections display and describe the graphics used in
		the gameplay elements, as well as the function of
		these elements
		% TODO Explain why we chose this graphics style
			\subsection{Laser target}\label{ssec:lasertarget}
			The laser target is the main target of the game. It consists of
			a small container, which contains a crystal. The point of the
			game is to direct a laser beam from an emitter to this target.
			When the target is hit by a laser beam, the outer columns around
			the crystal inside will rotate and spread out, indicating that
			the target has been hit. The game will then proceed to the
			next level.
			% TODO image.
			\subsection{Mirror}\label{ssec:mirror}
			A mirror is a crucial game element. Its reflective surfaces
			allow it to reflect any laser beam that hits these surfaces.
			It is also the only element that players can move and/or rotate.
			All levels require at least one mirror to move or rotate
			in order to hit the target. 
			% TODO image.
			\subsection{Wall}\label{ssec:wall}
			The wall is the main obstacle in the game. It blocks incoming
			laser beams completely. Walls are used in levels to make it 
			less easy for one player to reflect a laser beam coming from
			an emitter to the target.
			\subsection{Emitter}\label{ssec:emitter}
			The emitter is the most important aspect of the entire game.
			It is the only "true" source of a laser beam (although game
			elements like the beam splitter can also create beams, these
			elements always require input in the form of another laser
			beam; the emitter does not have that problem, hence it is a "true"
			source). In the early levels, players only move and rotate mirrors
			to guide a laser beam from the emitter to the target, while in the
			later levels beams have to be guided towards other game elements
			(like the beam splitter, for example) in order to complete the
			level. It is possible to have multiple emitters in a single level,
			and later levels use this to create more complex puzzles.