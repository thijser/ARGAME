\chapter{Design} \label{cha:design}
	This chapter explains the design of the system. This includes: back-end
	design, model/graphical design, and main activities.

	\section{Main activities} \label{sec:mainactivities}
		There are several main activities in the system, corresponding to the two
		main user types of the system. These user types are both the physically co-
		located	players as well as the physically remote players. The activities
		are as follows:

		\subsection{User wants to start a game} \label{ssec:userstartgame}
			A user (or player) wants to start a new game. This user can be either
			a physically co-located player or a physically remote player. In
			order to start a new game, the user has to click the correct buttons
			on the start screen. Clicking the buttons can only happen on a laptop,
			but that should not be a problem considering the user needs to be near
			their laptop anyway to view the game world from an Oculus Rift or a
			META One, the two tools required to play the game with. When the game
			is started, other users, both physically co-located and physically
			remote, can join the same session to play the game together.

		\subsection{Physically remote user wants to join a started game}
				\label{ssec:remotejoingame}
			A physically remote player wants to join an active game. Before
			this can happen, a game has to be started first. As mentioned
			before, both physically co-located players and physically remote
			players can start a game session which other players can join.

	\section{Back-end design} \label{sec:backenddesign}
		% TODO Unity, C#, MonoDevelop (maybe?)


	\section{Graphical design} \label{sec:graphicaldesign}
		For designing the 3D models, we used Blender. Blender is a free application
		for 3D modeling, and Unity natively support Blender models (provided Blender
		is installed on the system).

		% TODO Explain why we chose this graphics style
